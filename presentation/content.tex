\begin{comment}
  \begin{enumerate}
    \item Tematika (svarbiausių tematikos sąvokų apibrėžimai,
      įvadas į tematiką) - 1 skaidrė
    \item Problemos/uždavinio formulavimas (iš to turi išplaukti
      pagrindinė darbo tema) - 1 skaidrė
    \item Darbo tikslo formulavimas (jei sutampa su problemos
      formulavimu, tai kartoti nereikia) - 1 skaidrė
    \item Darbo planas (kaip pasieksite tikslą) - čia surašote
      punktus, kuriais buvo/bus pasiektas darbo tikslas - 1 skaidrė
    \item Temos gynimui (magistrantams): svarbiausios sąvokos
      (apibrėžimai, jei ilgi, neturi būti rašomi - turi būti sakomi
      žodžiu) - 2-3 skaidrės
    \item Darbo gynimui (bakalaurams, magistrantams): darbo metu
      iškilusios problemos, esminiai priimti sprendimai, kas buvo
      padaryta jūsų pačių ("susipažinau", "sukonspektavau",
      "išsiaiškinau", "išmokau" tinka tik kursiniams darbams!) - 3-6
      skaidrės
    \item (Laukiami) rezultatai - sąrašo pavidalu, ne daugiau trijų.
      "Susipažinau", "sukonspektavau", "išsiaiškinau", "išmokau" tinka
      tik kursiniams darbams! - 1 skaidrė
    \item Išvados - sąrašo pavidalu, ne daugiau trijų (temos gynime
      nereikia) - 1 skaidrė. Nebandykite rezultatų, pastebėjimų,
      komentarų, nuomonių apiforminti kaip išvadų!
  \end{enumerate}
\end{comment}

\begin{frame}
  \frametitle{Tematika}
  Komponentinis programavimas – tai programų sistemų kūrimo būdas,
  surenkant jas iš atskirų dalių (komponentų). Jo žadamos savybės:
  \begin{enumerate}
    \item Modifikuojamumas.
    \item Praplečiamumas.
    \item Perpanaudojamumas.
  \end{enumerate}
\end{frame}

\begin{frame}
  \frametitle{Problema}
  Martinas Odersky:
  \begin{enumerate}
    \item Komponentinių technologijų plitimas stringa.
    \item Priežastis yra dabartinių programavimo kalbų trūkumai.
    \item Scala – kalba, su kuria galima programuoti komponentiškai.
  \end{enumerate}
\end{frame}

\begin{frame}
  \frametitle{Darbo tikslas}
  Patikrinti ar Scala kūrėjų teiginys, jog su Scala yra įmanoma
  programuoti komponentiškai, yra teisingas.
\end{frame}

\begin{frame}
  \frametitle{Uždaviniai}
  \begin{enumerate}
    \item Apibrėžti, kas yra objektinis programavimas.
    \item Išsiaiškinti, kas yra komponentinis programavimas.
    \item Patikrinti, ar su Scala galima programuoti komponentiškai.
    \item Nustatyti Scala, kaip komponentinės technologijos, privalumus.
  \end{enumerate}
\end{frame}

TODO: Komponento savybės.
TODO: Komponentinis modelis.
TODO: Debian komponentinis modelis.
TODO: Scala komponentinis modelis.
TODO: Scala komponentinio išskirtinumai.

\begin{frame}
  \frametitle{Rezultatai}
  \begin{enumerate}
    \item Išskirtos esminės komponento savybės.
    \item Parodyta, kad Scala programavimo kalba ir Debian paketų
      tvarkymo sistema yra komponentinės technologijos.
    \item Nustatyta, kad pagrindinis Scala komponentinio modelio
      privalumas yra jo paprastumas.
  \end{enumerate}
\end{frame}

\begin{frame}
  \frametitle{Išvados}
  \begin{enumerate}
    \item Praplečiamumas ir modifikuojamumas: Scala.
    \item Perpanaudojamumas: Debian.
    \item Visos trys(?):
      \begin{enumerate}
        \item Visų komponento savybių realizavimas be išimčių.
        \item Platinimo mechanizmo apibrėžimas.
      \end{enumerate}
  \end{enumerate}
\end{frame}

\begin{frame}
  \frametitle{Klausimai?}
\end{frame}

\section{Probleminė situacija}

% Apie ką ir kodėl kalbame.

\begin{frame}
  \frametitle{Probleminė situacija}
  %\framesubtitle{TODO}
  Dėl poreikio kurti vis didesnes verslo palaikymo sistemas yra
  reikalingos priemonės, kurios:
  \begin{enumerate}
    \item Leistų kurti dideles sistemas evoliuciniu būdu.
    \item Leistu darbus efektyviai paskirstyti didelei grupei
      žmonių.
  \end{enumerate}
  \begin{comment}
    Kadangi verslo palaikymo sistemos perima vis daugiau verslo
    funkcijų, tai jos tampa vis didesnės ir todėl didėja poreikis
    turėti priemones, kurios:
    \begin{enumerate}
      \item Leistų kurti dideles sistemas evoliuciniu būdu, kas
        idealiu atveju būtų, kad papildant sistemą naujomis funkcijomis
        nereikia lįsti prie jau esamų realizacijos.
      \item Leistų sistemų kūrimo darbus efektyviai paskirstyti didelei
        grupei žmonių.
    \end{enumerate}
  \end{comment}
\end{frame}

\begin{frame}
  \frametitle{Komponentinis}
  \begin{enumerate}
    \item Modifikuojamumas.
    \item Praplečiamumas.
    \item Perpanaudojamumas.
  \end{enumerate}
  \begin{comment}
    Clemens Szyperski bei keleto kitų autorių teigimu sistemas kuriant
    komponentiškai, jos būtų lengviau modifikuojamos ir praplečiamos
    bei dėl to, kad komponentus nekeičiant galima naudoti keliose
    sistemose, didėja perpanaudojamumas.
  \end{comment}
\end{frame}

\begin{frame}
  %\frametitle{Scala}
  Martinas Odersky:
  \begin{enumerate}
    \item Komponentinių technologijų plitimas stringa.
    \item Priežastis yra dabartinių programavimo kalbų trūkumai.
    \item Scala – kalba, su kuria galima programuoti komponentiškai.
  \end{enumerate}
  \begin{comment}
    Martino Odersky teigimu komponentinių technologijų plitimas stringa,
    nes dabartinėse programavimo kalbose nėra realizuotos priemonės
    leidžiančios apibrėžti ir sujungti komponentus. Jo siūlomas
    sprendimas: programavimo kalba Scala, su kuria galima programuoti
    komponentiškai.
  \end{comment}
\end{frame}

\section{Darbo tikslas}

% Ko buvo siekta atliekant darbą.

\begin{frame}
  \frametitle{Tikslas}
  Patikrinti ar Scala kūrėjų teiginys, jog su Scala yra įmanoma
  programuoti komponentiškai, yra teisingas.
\end{frame}

\begin{frame}
  \frametitle{Uždaviniai}
  \begin{enumerate}
    \item Apibrėžti, kas yra objektinis programavimas.
    \item Išsiaiškinti, kas yra komponentinis programavimas.
    \item Patikrinti, ar su Scala galima programuoti komponentiškai.
    \item 
  \end{enumerate}<++>
\end{frame}<++>

\section{Darbo rezultatai}

\section{Darbo išvados}
