\documentclass[12pt, a4paper, xcolor=table]{documentation}

\usepackage{style}
\usepackage{xparse}
\usepackage{pdfcomment}
\usepackage{caption}
\usepackage{environ}
\usepackage{fancyvrb}
\usepackage{scalatools}

%\renewcommand{\listingscaption}{kodo fragmentas}
\renewcommand{\listoflistingscaption}{Fragmentų sąrašas}
%\renewcommand{\thelisting}{\arabic{listing}}
\DeclareCaptionLabelFormat{ltcaptionlisting}{#2 fragmentas}
\captionsetup[listing]{labelformat=ltcaptionlisting}

\newcounter{glossaryCounter}
\makeatletter
\def\glossaryLabel#1{\begingroup
  \def\@currentlabel{\textsuperscript{Ž\arabic{glossaryCounter}}}%
	\phantomsection\label{#1}\endgroup
}
\makeatother
\renewenvironment{glossary}{%
\setcounter{glossaryCounter}{0}%
\NewDocumentEnvironment{entry}{m m o o}{%
\addtocounter{glossaryCounter}{1}%
\glossaryLabel{glossary:##1}%
{\noindent Ž\arabic{glossaryCounter} \strong{##2}} \par
}{%
  \IfNoValueTF{##3}{%
  }{%

  Angliškai: \emph{##3}
  }
  \IfNoValueTF{##4}{%
  }{%

  Aprašymas: ##4
  }
}
}{%
}
\newcommand{\gls}[2]{%
\pdfmarkupcomment[markup=Highlight]{#2}{Paaiškinimas.}
\ref{glossary:#1}%
}
\pdfcommentsetup{color={0.8 0.8 0.8}}

\newcommand{\TODO}[1]{} %{TODO: \emph{#1}}
\newcommand{\FIXME}[1]{} %{FIXME: \emph{#1}}
\newcommand{\NFIXME}[1]{} %{\footnote{FIXME: #1}}
\newcommand{\NTODO}[1]{} %{\footnote{TODO: #1}}
\newcommand{\varname}[1]{\strong{#1}}
\newcommand{\plangname}[1]{\strong{#1}}
\newcommand{\progname}[1]{\strong{#1}}
\newcommand{\human}[1]{#1}
\newcommand{\inlinequote}[1]{\emph{#1}}

\begin{document}
  % Dokumento pavadinimas.
\docname{Komponentinis programavimas su Scala}
\docnameen{Component-based programming with Scala}

% Dokumento tipas.
\doctype{Kursinis darbas}

% Informacija apie autorių.
\authorname{Vytautas Astrauskas}
\coursenumber{3}
\groupnumber{2}

% Informacija apie vadovą.
\supervisorname{lektorius Donatas Čiukšys}

  \maketitle
  {
    \singlespacing
    \tableofcontents
  }
  
  \pagestyle{plain}
  %\begin{abstract}
  TODO: Parašyti.
\end{abstract}

  \chapter{Įvadas}

\section{Temos aktualumas}

Vis didėjant sistemoms, jų vystymas darosi vis sudėtingesnis procesas.
Dėl šios priežasties auga poreikis turėti priemones, kurios:
\begin{itemize}
  \item leistų sistemas kurti evoliuciniu būdu, jas vis papildant nauju
    funkcionalumu neliečiant jau veikiančių sistemos dalių;
  \item leistų sistemos kūrimo darbą efektyviai paskirstyti dideliam
    kiekiui žmonių.
\end{itemize}
TODO: Susieti su komponentiniu programavimu.

\section{Darbo tikslas}

Straipsnyje \cite{scalable-component-abstractions} Martin Odersky
ir Matthias Zenger teigia, jog su Scala galima programuoti
komponentiškai. Šio darbo tikslas yra patikrinti ar jų teiginys yra
teisingas ir jei taip, tai kokias būtent papildomas galimybes
suteikia komponentinis programavimas lyginant su statiniu
klasiniu programavimu (kaip pavyzdys nagrinėjama Java) kuriant verslo
informacines sistemas.

TODO: Martin Odersky nurodyti dabartinių programavimo kalbų trūkumai.

\section{Siekiami rezultatai}

Uždaviniai:
\begin{enumerate}
  \item išsiaiškinti kas yra objektinis programavimas;
  \item išsiaiškinti kas yra komponentinis programavimas;
  \item apibrėžti komponento sąvoką;
  \item \label{task:enum:scala-component} patikrinti ar su Scala galima
    programuoti komponentiškai;
  \item jei \ref{task:enum:scala-component}, tai nustatyti kokios
    būtent Scala galimybės tai leidžia;
  \item išsiaiškinti, kokias papildomas galimybes suteikia Scala
    lyginant su Java ir kaip turėtų būti programuojama, kad jos
    būtų išnaudojamos.
\end{enumerate}

TODO: Paaiškinti uždavinių ryšius: kodėl yra svarbu išsiaiškinti kas
yra objektinis programavimas.

\chapter{Objektinis programavimas}

\section{Objektinio programavimo apibrėžimai}

TODO:
\begin{itemize}
  \item Objektinio programavimo apibrėžimas:
    \begin{itemize}
      \item dinaminis susiejimas;
      \item inkapsuliacija;
      \item potipiai;
      \item paveldėjimas, pavedimas;
      \item atvira rekursija.
    \end{itemize}
  \item Objektinio programavimo rūšys ir kur patartina ką naudoti:
    \begin{itemize}
      \item prototipinis objektinis;
      \item dinaminis klasinis objektinis;
      \item statinis klasinis objektinis.
    \end{itemize}
\end{itemize}

\section{Statinis klasinis objektiškai orientuotas programavimas}

TODO: Apibrėžimas ir nurodymas, kad toliau darbe apsiribojama tik juo.
Kaip pagrindinis pavyzdys nagrinėjama Java programavimo kalba.

\chapter{Komponentinis programavimas}

\section{Komponento apibrėžimai}

TODO: Įvairių autorių pateikti komponento apibrėžimai.

\section{Komponentinio modelio apibrėžimas}

TODO: Komponentinio programavimo supratimo pokytis, perėjimas nuo
bandymo apibrėžti kas yra komponentas prie komponentinio modelio
sąvokos. Komponentinio modelio apibrėžimas pagal
\cite{classification-framework-for-scm}. Paaiškinimas, kuo svarbus šis
pokytis (Clemens Szyperski nurodo, kad komponentas yra diegimo
vienetas\cite{cs-beyond-object-oriented-programming}, bet tiesiog
pasakymas, kad komponentas turi būti įdiegiamas (TODO: kritika
išsakyta Szyperski'ui dėl jo minties, kad komponentas turi būti
sukompiliuotas) duoda mažai naudos, tuo tarpu komponentinis
modelis apibrėžia kaip yra platinami ir įdiegiami komponentai).

\section{Komponento savybių sąrašas}

TODO: Iš ankstesnių skyrelių surinktas komponento būtinųjų savybių
sąrašas.

\chapter{Scala}

\section{Scala ir Java skirtumai}

TODO: Ką Scala pridėjo objektiniam (Java):
\begin{itemize}
  \item Scala tipų sistema;
  \item Scala savybės \en{trait} konstrukcija;
  \item savybių jungimas į klases panaudojant modulių maišos
    kompoziciją \en{modular mixin composition};
  \item atviros rekursijos anotacija \en{selftype annotations} –
    iš principo tik sintaksinis saldainiukas;
  \item savybių naudojimo vietoj klasių privalumai, pavyzdys e1
    (savybės perdengimas vidury hierarchijos).
\end{itemize}

TODO: Scala vidinių klasių ir Java įdėtinių klasių skirtumai. Taip pat
pavyzdys su Scala tipų sistema ir Abstract factory projektavimo šablonu.

TODO: Scala savybės nuo Java sąsajų skiriasi tuo, kad savybės gali turėti
metodų realizacijas.

TODO: Scala savybės nuo klasių skiriasi tuo, kad savybės negali turėti
konstruktoriaus parametrų. Šis apribojimas leidžia efektyviai
išspręsti su multipaveldėjimu susijusią rombo problemą.

Šio skyrelio tikslas parodyti kuo Scala papildė Java objektinį ir
kokias papildomas galimybes tai suteikia. (Scala, kaip komponentinė
programavimo kalba dar nenagrinėjama)

\section{Scala komponentinis modelis}

TODO: Scala komponentinio modelio apibrėžimas (pagal išrinktas savybes).
Parodymas, kad Scala komponentinis modelis tenkina reikalavimus,
nurodymas kokios būtent Scala programavimo kalbos savybės kuriuos
būtent reikalavimus realizuoja.

Scala komponentinio modelio privalumai (remiantis
\cite{classification-framework-for-scm}):
\begin{itemize}
  \item vertikalaus sujungimo\cite[599]{classification-framework-for-scm}
    palaikymas leidžia kurti įvairaus dydžio \en{scalable} komponentus.
\end{itemize}

Scala komponentinio modelio trūkumai:
\begin{itemize}
  \item nėra komponentų, kaip paketų, valdymo mechanizmo (šitai galima
    apeiti pasinaudojant versijų kontrolės sistemų galimybėmis,
    pavyzdžiui, \verb|git submodule|);
  \item komponentų kiekis turi būti žinomas projektavimo metu, todėl
    Scala komponentinis modelis nėra tinkamas, pavyzdžiui,
    grafinės sąsajos elementų bibliotekos \en{GUI widgets toolkit}
    kūrimui. (Kadangi komponentai yra „užregistruojami“
    kompiliavimo metu, tai nėra, kaip pridėti trečiųjų šalių
    komponento realizacijos.)
\end{itemize}

\section{Pavyzdys: Scala komponentinis prieš Java objektinį}

e6 analizė. Cake pattern taikymas: turime ComponentA ir ComponentB,
juos sujungę į System gauname sistemą sudarytą iš dviejų komponentų.
Norėdami realizuoti analogišką sistemą su Java, turime du variantus:
\begin{enumerate}
  \item visą kodą sudėti į System, bet tada bus neįmanomas kodo
    perpanaudojimas, taip pat bus sudėtingesnis sistemos kūrimo
    darbų paskirstymas keliems programuotojams;
  \item \label{scala:exmp:enum:2} paveldėjimą pakeisti į delegavimą,
    tuo pačiu sukuriant Container, kuris pasirūpintų dalių
    sujungimu. Šiuo atveju vėl gauname kažką panašaus į
    komponentinį.
\end{enumerate}

Šis pavyzdys iliustruoja teiginį, kad komponentinis programavimas yra
susijęs ne su realizacine technologija (programavimo kalba), o su
jos naudojimo būdu, kuris užtikrina:
\begin{enumerate}
  \item galimybę dalis kurti atskirai, nepriklausomai viena nuo kitos;
  \item galimybę lengvai dalį pakeisti kita jos realizacija.
\end{enumerate}

\chapter{Išvados}

Šiame darbe buvo apibrėžta kas yra komponentinis modelis ir parodyta,
jog Scala autorių teiginys, jog su Scala galima programuoti
komponentiškai yra teisingas.

\chapter{Neišnagrinėti klausimai}

Komponentų vykdymo aplinka. Komponento konteineris. Galimybė vieną
komponentą pakeisti kitu vykdymo metu bei su tuo susijęs komponento
gyvavimo ciklas. (Pagal OSGi.)

Įtaisytųjų \en{embedded} sistemų komponentų ir paketinės sistemos
paketų (pavyzdžiui, Debian DEB) skirtumai.

Programuojant bet kaip su Java, tai tikimybė, kad gausis kažkas artimo
taisyklingam komponentiniam yra labai maža. Programuojant bet kaip su
Scala turėtų gautis žymiai mažesnis $\Delta$. \emph{Jeigu programuodami
su Java naudojame projektavimo šablonus, tai tikėtina, kad gausime
sistemą, kurią galime nesunkiai išskaidyti į komponentus. Norint
programuoti komponentiškai su Scala irgi reikia taikyti projektavimo
šablonus (Cake pattern, Stackable trait). Kitaip tariant mažai tikėtina,
kad programuojant bet kaip gausime kažką panašaus į komponentinį.}

Kaip pagerinti Java modifikuojamumą. \emph{Nenaudojant komponentinio,
modifikuojamumą galimą padidinti klasių hierarchiją pakeičiant
į savybių hierarchiją.}

Tiek objektiniame, tiek komponentiniame dažnai gauname, kad dalykinės
srities kodas yra „paslepiamas“ „palaikančio“ kodo (žurnalizavimas,
tranzakcijos, saugumas ir t.t.). Objektiniame šitai bandoma spręsti
naudojant Aspect-oriented programming. Kaip šitai yra bandoma
spręsti komponentiniame (pavyzdžiui, EJB CDI Portable extensions)
ir kokie tokių sprendimų privalumai lyginant su AOP?

\Chapter{Objektinis programavimas}

Norint atskirti objektinę paradigmą nuo komponentinės, reikia
tiksliai apibrėžti, kas priklauso kiekvienai iš jų.
Šiame skyriuje pabandyta išskirti, kas yra objektinis programavimas
apskritai, kokios yra objektinio programavimo rūšys bei kam
tos rūšys yra reikalingos (jų pritaikymo sritys).

\TODO{%
Nagrinėjant Scala yra svarbu nurodyti ant kokio objektinio ji
„stovi“. (Tam, kad pagrįsti kodėl nurodau objektinio rūšis.)
%
Nagrinėjant Scala panagrinėti ar ji keičia statinio klasinio
objektinio pritaikymo sritį.
}

\section{Objektinio programavimo apibrėžimas}

\cite[225]{types-and-programming-languages} teigimu yra beprasmiška
bandyti tiksliai apibrėžti, ką reiškia „objektiškai orientuotas“,
bet objektinį programavimą bendrąja prasme galima suvokti, kaip
programavimą pasinaudojant objekto abstrakcija. Objektą galima
apibrėžti, kaip \inlinequote{esybę, kuri apjungia duomenų ir procedūrų
savybes tam, kad galėtų atlikti veiksmus ir turėti būseną}
\cite[41]{OOP-themes-and-variations}.  Objektinės sistemos veikimas
yra suvokiamas, kaip objektų tarpusavio sąveika: objektai vienas kitam
siunčia žinutes, o kiekvienas gavęs žinutę objektas į ją sureaguoja
atlikdamas atitinkamą operaciją
\cites[41]{OOP-themes-and-variations}%
[277]{concepts-in-programming-languages}%
[168]{Wegner:1987:DOL:38807.38823}.
Objektų būsena paprastai yra realizuojama objektų viduje saugomais
duomenimis, o žinučių siuntimas – su objektu susietų procedūrų,
kurios vadinamos metodais, kvietimu \cite[41]{OOP-themes-and-variations}.
Be būsenos ir operacijų, prie objekto išskirtinių savybių
\cite[37]{cs-beyond-object-oriented-programming} bei populiariausių
objektinių programavimo kalbų kūrėjai nurodo dar ir tai, kad
kiekvienas objektas turi \gls{object-identity}{unikalią tapatybę}.
\cite[37]{cs-beyond-object-oriented-programming} nurodo,
kad \gls{object-identity}{objekto tapatybė} yra tai, kas leidžia jį
išskirti iš kitų, nepriklausomai nuo to, kokius pokyčius patiria pats
objektas, bei kaip pavyzdį pateikia istoriją apie Džordžo Vašingtono
kirvį, kuris turėjo penkis naujus kotus ir keturias galvas, bet vis
tiek visą laiką išsaugojo tą pačią tapatybę – buvo Džordžo
Vašingtono kirvis. Taigi apibendrinant, objektai yra esybės, kurios:
\begin{enumerate}
  \item Turi \gls{object-identity}{unikalią tapatybę}.
  \item Turi būsenas.
  \item Sąveikauja besikeisdamos žinutėmis.
\end{enumerate}

Siekiant išskirti
\gls{object-oriented-programming-language}{objektines programavimo
kalbas} iš programavimo kalbų, kuriomis įmanoma programuoti
objektiškai (pavyzdžiui, tarę, kad programavimo kalbos \plangname{Modula-2}
modulis yra objektas, su ja galėtume programuoti objektiškai), yra
reikalaujama, kad programavimo kalba, kurią vadiname objektine
palaikytų objektinį programavimą kalbinėmis priemonėmis. 1987 metų
ACM\footnote{„Association for Computing Machinery“ yra viena didžiausių
ir prestižiškiausių mokslinių ir mokymo bendruomenių, kurių domėjimosi
sritis yra kompiuteriniai skaičiavimai.}
OOPSLA\footnote{„Objektiškai orientuotas programavimas, sistemos, kalbos
ir taikomosios programos“ \en{Object-Oriented Programming, Systems,
Languages \& Applications} yra kasmetinė ACM tyrimų konferencija.}
konferencijos metu paskelbtame „Orlando
susitarime“\footnote{1987 metų ACM OOPSLA konferencija vyko Orlando
mieste, Floridoje, JAV.}
\en{„The Treaty of Orlando“}
\cite{Lieberman:1987:TO:62139.62144} nurodoma, kad esminis
objektinių programavimo kalbų bruožas yra 
\gls{object-description-sharing-mechanism}{dalinimosi \en{sharing}
mechanizmas} – galimybė perpanaudoti esamų objektų apibrėžimų dalis
apibrėžiant naujus objektus.
\gls{object-oriented-programming-language}{Objektinės programavimo kalbos}
tai gali įgyvendinti panaudodamos \gls{delegation}{delegavimo} arba
\gls{class-inheritance}{paveldėjimo}
mechanizmus. Šis mechanizmas yra itin svarbus tuo, kad leidžia
objektus apibrėžti palaipsniui. Taigi, \plangname{Modula-2} nėra
objektinė programavimo kalba, nes jos modulis negali automatiškai
„perimti“ dalies kito modulio veiksenos.

Be šios „Orlando susitarime“ nurodytos savybės, dauguma
objektinių programavimo kalbų turi ir daugiau bendrų bruožų,
kuriuos išskyrė \cite[225-227]{types-and-programming-languages}:
\begin{enumerate}
  \item \emph{\gls{dynamic-dispatch}{Dinaminis susiejimas}} –
    objekto reakcija į gautą žinutę yra nustatoma vykdymo metu.
  \item \emph{\gls{encapsulation}{Uždarumas}} – vidinė objekto
    struktūra yra slepiama.
  \item \emph{\gls{subtyping}{Potipiai}} – kai norime pasinaudoti objektu, 
    tai mums rūpi tik jo sąsaja ir mes galime naudoti objektą $I$ vietoj
    objekto $J$, jei objekto $J$ sąsaja yra objekto $I$ sąsajos poaibis.
  \item \emph{\gls{object-description-sharing-mechanism}{Paveldėjimas,
    pavedimas}} – galimybė perpanaudoti jau egzistuojantį kodą: tai
    gali būti pasiekiama objektų kūrimui naudojant klases, kurios
    gali paveldėti kai kurias savybes iš tėvinių klasių, arba
    naudojant žinučių persiuntimą.
  \item \emph{\gls{open-recursion}{Atvira rekursija}} – specialaus
    kintamojo (dažniausiai jis vadinamas \varname{this}, arba
    \varname{self}) egzistavimas, kuriuo pasinaudojant galima kreiptis
    į kitus to paties objekto metodus.
\end{enumerate}
Nors paminėtos savybės ir yra bendros populiariausioms objektinėms
programavimo kalboms, tačiau šias savybes jos įgyvendina įvairiais
būdais, dėl ko įgauna savitumų, kurie daro įtaką šių
programavimo kalbų galimoms pritaikymo sritims.

\section{Objektinių kalbų rūšys}

Objektinės programavimo kalbos gali būti labai plačiai naudojamos:
tiek eksperimentiniam, tiriamajam bei prototipiniam programavimui tiek
jau produkcijai skirtoms sistemoms. Pirmuoju atveju paprastai dalykinė
sritis yra prastai žinoma, todėl yra itin svarbus programavimo kalbos
lankstumas – galimybė greitai atlikti pakeitimus. Antruoju atveju
svarbesniu tampa sistemos patikimumas
\cite{Lieberman:1987:TO:62139.62144}. Objektinės programavimo kalbos
lankstumui / stabilumui įtaką daro jos 
\gls{object-description-sharing-mechanism}{dalinimosi mechanizmo}
savybės \cite{Lieberman:1987:TO:62139.62144} bei jos naudojamos
tipų sistemos savybės.

\cite{Lieberman:1987:TO:62139.62144} teigimu lankstumą programavimo
kalbai suteikia šios trys jos naudojamo
\gls{object-description-sharing-mechanism}{dalinimosi mechanizmo}
savybės:
\begin{enumerate}
  \item Galimybė modifikuoti atskiro objekto veikseną – pavyzdžiui,
    vienam konkrečiam objektui pridėti naują metodą.
  \item Galimybė keisti
    \gls{object-description-sharing-mechanism}{dalinimosi
    mechanizmo} ryšius vykdymo metu – pavyzdžiui, pakeisti tėvinį
    objektą.
  \item Galimybė išreikštinai nurodyti
    \gls{object-description-sharing-mechanism}{dalinimosi mechanizmo}
    ryšius – pavyzdžiui, kokios žinutės kokiam objektui yra
    persiunčiamos.
\end{enumerate}
Šiomis savybėmis pasižymi
\gls{prototype-based-programming-language}{prototipinės objektinės
programavimo kalbos}, iš kurių, turbūt, populiariausia atstovė yra
\plangname{JavaScript}.

\cite{Lieberman:1987:TO:62139.62144} teigimu patikimumą padeda
užtikrinti šios \gls{object-description-sharing-mechanism}{
dalinimosi mechanizmo} savybės:
\begin{enumerate}
  \item Galimybė užtikrinti, kad visi konkrečiai grupei priklausantys
    objektai elgsis vienodai. Paprastai tai yra užtikrinama
    pasinaudojant klasės konstrukcija.
  \item Draudimas keisti objekto apibrėžimą po jo sukūrimo – tai
    gali būti naudinga, pavyzdžiui, tuo, kad yra garantuojama,
    jog objektas tenkina tą patį kontraktą, kurį jis tenkino iš
    karto po sukūrimo.
  \item Vieningas \gls{object-description-sharing-mechanism}{
    apibrėžimų dalinimosi mechanizmas}, kuris palengvina supratimą
    kaip veikia sistema.
\end{enumerate}
Šiomis savybėmis pasižymi
\gls{class-based-programming-language}{klasinės objektinės
programavimo kalbos}, tokios, kaip \plangname{Java}, \plangname{C++},
\plangname{Python}, \plangname{Ruby}.

Kaip pavyzdys situacijos, kur prototipinės objektinės programavimo
kalbos savybės yra naudingesnės nei klasinės objektinės, galėtų
būti natūralios žmonių kalbos sintezatoriaus kūrimas. Tarkime, kad
mums reikia galimybės turint veiksmažodžio bendratį susigeneruoti
visas jo formas. Naudodami klasinę objektinę programavimo kalbą
kiekvienai veiksmažodžių rūšiai galėtume susikurti po klasę,
kuri gavusi bendratį mokėtų sugeneruoti tos rūšies veiksmažodžių
formas. Problema ta, kad natūralios žmonių kalbos pasižymi didele
išimčių gausa ir šiuo atveju kiekvienai išimčiai irgi reikėtų
sukurti po klasę. Jei naudotume prototipinę objektinę programavimo
kalbą, tai kiekvieną kartą pridėdami po naują veiksmažodžio
objektą, jam kaip prototipą galėtume nurodyti panašiausią į jį ir
tereikėtų perrašyti tik jų skirtumus.

Be \gls{object-description-sharing-mechanism}{dalinimosi mechanizmo},
objektinių programavimo kalbų lankstumui ir patikimumui įtaką
dar daro jų naudojamos tipų sistemos. \cite[2]{Madsen:1990:STO:97946.97964}
teigimu didesniu lankstumu pasižymi
\gls{weakly-typed-programming-language}{silpnai tipizuotos} bei
naudojančios \gls{dynamically-typed-programming-language}{dinaminį
tipų tikrinimo mechanizmą} kalbos. Tuo tarpu
\gls{strongly-typed-programming-language}{stipri tipizacija} ir
\gls{statically-typed-programming-language}{statinis tipų tikrinimo
mechanizmas} padeda užtikrinti programų patikimumą, nes tai
leidžia nemažai klaidų rasti dar programos kompiliavimo stadijoje.

Apibendrinant galima būtų teigti, kad kuriant sistemas, kurioms yra
itin svarbus patikimumas, reikėtų naudoti
\gls{statically-typed-programming-language}{statines}
\gls{strongly-typed-programming-language}{stipriai tipizuotas}
\gls{class-based-programming-language}{klasines} programavimo kalbas.
Kadangi šiame darbe yra nagrinėjamos priemonės, kurios pagelbėtų
kuriant verslo palaikymo sistemas, kurioms ir yra itin svarbus
patikimumo kriterijus, tai toliau darbe yra nagrinėjamos tik
\gls{statically-typed-programming-language}{statinės}
\gls{strongly-typed-programming-language}{stipriai tipizuotos}
\gls{class-based-programming-language}{klasinės}
\gls{object-oriented-programming-language}{objektinės} programavimo
kalbos.

\chapter{Komponentinis programavimas}

Šiame skyriuje yra surinkti įvairių autorių pateikiami komponento
apibrėžimai, pateikiamas komponento savybių sąrašas, pristatoma kas
yra komponentinis modelis, nurodomi objektinio ir komponentinio
skirtumai.

\section{Komponento apibrėžimai}

Nors apie programinės įrangos komponentus yra kalbama jau daugiau
nei dešimt metų, vis dar nėra vieningo apibrėžimo, kas per esybė
yra komponentas\cite{classification-framework-for-scm}. Vienas
iš senesnių ir, turbūt, dažniausiai cituojamų yra Clemens Szyperski
\cite{cs-beyond-object-oriented-programming}
pateiktas komponento apibrėžimas\footnote{Google duomenimis pacituotas
daugiau nei 6 tūkstančius kartų.}:
\begin{quote}
  Programinės įrangos komponentas yra kompozicijos elementas su
  sutartinai apibrėžtomis sąsajomis ir tik su išreikštinai
  nurodytomis priklausomybėmis. Programinės įrangos komponentas
  gali būti naudojamas nepriklausomai bei panaudotas
  kompozicijoje trečiųjų šalių.
\end{quote}
Nors apibrėžime to išreikštinai ir nėra, tačiau kaip vieną iš
esminių komponento savybių autorius nurodo, kad komponentas yra
nepriklausomo diegimo vienetas \en{unit of deployment}
\cite[36]{cs-beyond-object-oriented-programming} ir netgi
pareikalauja, kad komponentai klientui būtų pateikiami jau
sukompiliuoti \cite{point-counterpoint}. Kitaip tariant, komponentas
yra laikomas fiziniu paketu\footnote{Fizinis paketas nuo loginio paketo
skiriasi tuo, kad fiziniam paketui turi būti įmanoma įvykdyti
operaciją „perkelti iš vieno kompiuterio į kitą“.}. Ši
komponento savybė yra išreikštinai
nurodyta \cite[385]{objects-components-and-frameworks-with-uml}
apibrėžime:
\begin{quote}
  Komponentas (kode) yra rišlus programinės įrangos realizacijos
  paketas, kuris:
  \begin{enumerate}
    \item gali būti nepriklausomai kuriamas ir pristatytas;
    \item turi išreikštines ir gerai apibrėžtas sąsajas savo teikiamiems
      servisams;
    \item turi išreikštines ir gerai apibrėžtas sąsajas servisams,
      kurių jis tikisi iš kitų;
    \item gali būti sujungtas su kitais komponentais, galbūt pritaikant
      kai kurias savybes \en{properties}, bet be pačių komponentų
      keitimo.
  \end{enumerate}
\end{quote}
bei \cite[1]{Hopkins:2000:CP:352183.352198} apibrėžime:
\begin{quote}
  Programinės įrangos komponentas yra fizinė vykdomosios programinės
  įrangos pakuotė su gerai apibrėžta ir vieša sąsaja.
\end{quote}

Su tuo, kad komponentas yra fizinis diegimo vienetas yra susiję
tai, kad jis yra uždaras
\cite[36]{cs-beyond-object-oriented-programming}. Kitaip tariant,
sistemos naudotojai į komponentą žiūri, kaip į juodą
dėžę\cite[2]{Gill:2003:CMF:966221.966237}.
Kadangi nėra galimybės pasižiūrėti kas yra komponento „viduje“,
tai norint pasinaudoti jo teikiamu funkcionalumu reikia, kad būtų
pateiktos išreikštinės sąsajos, per kurias galima prie jo
prieiti\cite[387]{objects-components-and-frameworks-with-uml}
\cite[36]{cs-beyond-object-oriented-programming}. Dėl tos
pačios priežasties tam, kad galėtume komponentui leisti pasinaudoti
jam reikiamais servisais, jie irgi turi būti išreikštinai nurodyti
\cite[387]{objects-components-and-frameworks-with-uml}
\cite[36]{cs-beyond-object-oriented-programming}. Visos šios
savybės leidžia komponentų kūrėjams kurti komponentus nieko
nežinant apie komponentų naudotojus\cite[2]{Gill:2003:CMF:966221.966237}
\cite[139]{meyer1999components}
kas lemia, kad sistema gali būti surinkta iš nepriklausomai sukurtų
komponentų. Taip pat tie patys komponentai be pakeitimų gali būti
panaudoti keliose skirtingose trečiųjų šalių sistemose
\cite[388]{objects-components-and-frameworks-with-uml}
\cite[36]{cs-beyond-object-oriented-programming}. Taigi
apibendrinant, galima būtų išskirti tokias komponento savybes:
\begin{enumerate}
  \item \label{com:exe:deployment} yra fizinis diegimo vienetas;
  \item \label{com:exe:blackbox} yra juoda dėžė;
  \item \label{com:exe:interfaceprovider} prie jo funkcionalumo galima
    prieiti tik per išreikštinai apibrėžtas sąsajas;
  \item \label{com:exe:interfaceuser} savo poreikius nurodo tik per
    išreikštinai apibrėžtas sąsajas;
  \item \label{com:exe:independent} gali būti kuriamas nepriklausomai;
  \item \label{com:exe:reusable} gali būti perpanaudojamas;
  \item \label{com:exe:composed} gali būti be pakeitimų sujungtas su
    kitais komponentais į vieną sistemą;
  \item \label{com:exe:interchangable} gali būti sistemoje
    pakeičiamas kitu, jei naujasis įgyvendina visą senojo
    funkcionalumą ir nereikalauja nieko daugiau.
\end{enumerate}

\section{Komponentinio modelio apibrėžimas}

Ankstesniame skyrelyje pateiktais komponento apibrėžimais pasakoma kas
yra komponentas, bet nėra pasakoma kaip yra kuriami komponentai ir kaip
iš jų sukomponuojama sistema. Šitai apibrėžia komponentinis
modelis. \cite[37]{heineman2001component} nurodo, kad
\begin{quote}
  Komponentinis modelis apibrėžia standartus komponentų realizacijai,
  naudojamiems vardams, tarpusavio sąveikai, tinkinimui, kompozicijai,
  evoliucijai ir įdiegimui.
\end{quote}
Pasinaudodami komponentinio modelio apibrėžimu autoriai šiek tiek
kitaip apibrėžia ir patį komponentą\cite[7]{heineman2001component}:
\begin{quote}
  Programinės įrangos komponentas yra programinis elementas kuris
  atitinka komponentinį modelį ir gali būti nepriklausomai įdiegtas
  ir jo nekeičiant įkomponuotas laikantis komponavimo standarto.
\end{quote}

Komponentinio programavimo nagrinėjimas akcentuojant komponentinį modelį
turi privalumą, kad remiantis juo galima nagrinėti skirtingas
komponentinių sistemų kūrimo metodikas bei tuo pačiu išskirti
kiekvienos iš jų privalumus ir trūkumus.
\cite[4]{classification-framework-for-scm} išskyrė tokias komponentinių
modelių klasifikavimo dimensijas:
\begin{description}
  \item[gyvavimo ciklas] – kiek komponentinis modelis paremia kiekvieną
    iš komponento gyvavimo ciklo stadijų;
  \item[konstravimas] – kokiu būdu yra sukonstruojama sistema iš
    komponentų;
  \item[ekstra-funkcinės savybės] – ką komponentinis modelis siūlo
    ekstra-funkcinių savybių specifikavimui, valdymui ir kompozicijai.
\end{description}

Šis komponento apibrėžimas, pasinaudojant komponentinio modelio
apibrėžimu, yra naudingas tuo, jog leidžia kalbėti apie
komponentais paremtą programų sistemų inžineriją.

Komponentinio modelio akcentavimas vietoje  
  

Bandant projektuoti komponentines sistemas, pasinaudojant komponento
sąvoka, iškyla problema, kad arba komponento apibrėžimas yra per
siauras ir „nepagauna komponentinio idėjos“ arba yra per platus ir
dėl to tampa sunkiai pritaikomas. Pavyzdžiui, nurodymas, kad komponentas
yra fizinis diegimo vienetas nieko nepasako apie tai, kaip

Tam, kad komponentus galima būtų sujungti į vieną sistemą, reikalinga,
kad komponentai būtų suderinami. 

Komponentinį modelį jau mini
\cite[42]{cs-beyond-object-oriented-programming}.

TODO: Komponentinio programavimo supratimo pokytis, perėjimas nuo
bandymo apibrėžti kas yra komponentas prie komponentinio modelio
sąvokos. Komponentinio modelio apibrėžimas pagal
\cite{classification-framework-for-scm}. Paaiškinimas, kuo svarbus šis
pokytis (Clemens Szyperski nurodo, kad komponentas yra diegimo
vienetas\cite{cs-beyond-object-oriented-programming}, bet tiesiog
pasakymas, kad komponentas turi būti įdiegiamas (TODO: kritika
išsakyta Szyperski'ui dėl jo minties, kad komponentas turi būti
sukompiliuotas) duoda mažai naudos, tuo tarpu komponentinis
modelis apibrėžia kaip yra platinami ir įdiegiami komponentai).

TODO: Pagrindimas, kad komponentinis ir objektinis yra skirtingi
dalykai.

\section{Pastabos}

Su Szyperski teiginiu, kad komponentai būtinai turi būti sukompiliuoti
Meyer iš principo nesutinka ir teigia, kad tai yra reikalinga tik
tiek, kiek padeda garantuoti informacijos slėpimą.
\url{http://web.archive.org/web/20010608234054/http://www.sdmagazine.com/articles/1999/9911/9911k/9911k.htm}

Pagal Szyperski, komponentų perpanaudojimas nėra pagrindinis argumentas
naudoti komponentus. Svarbiausia yra kuriamos sistemos praplečiamumas
\en{extensibility} ir modifikuojamumas \en{evolvability}.
(„Why use components“ iš
\url{http://web.archive.org/web/20010609003644/http://www.sdmagazine.com/articles/2000/0002/0002l/0002l.htm})

\section{Objektinio ir komponentinio skirtumai}

Paminėti, kad nors komponentai ir neprivalo būti kuriami panaudojant
objektinį, bet komponentus kurti pasinaudojant objektiniu yra
natūralu. (Pacituoti Bertrand Meyer ir
\cite[389]{objects-components-and-frameworks-with-uml},
\cite[11]{cs-beyond-object-oriented-programming},
\cite[36]{heineman2001component}.)

\chapter{Scala}

Šiame skyriuje pristatyta programavimo kalba \plangname{Scala}:
nurodyta kuo ji skiriasi nuo \plangname{Java}, aprašytas
ir išanalizuotas \plangname{Scala} komponentinis modelis,
išanalizuoti \plangname{Scala} privalumai lyginant su \plangname{Java}.

\section{\plangname{Scala} ir \plangname{Java} skirtumai}

\plangname{Scala} kūrėjų \cite[1]{scala-overview} teigimu bent iš
dalies komponentinių technologijų evoliucijai trukdo programavimo
kalbų, kurios yra naudojamos komponentų kūrimui ir jų jungimui,
trūkumai. Dėl šios priežasties jie pabandė sukurti programavimo
kalbą, su kuria naudojant tas pačias priemones galima būtų aprašyti
tiek mažas tiek dideles dalis\footnote{\plangname{Scala} išsišifruoja,
kaip \emph{scalable language}.}. \plangname{Scala} yra multiparadigminė
programavimo kalba, derinanti objektinio ir funkcinio programavimo
savybes:
\begin{itemize}
  \item ji yra objektinė kalba ta prasme, kad kiekviena reikšmė
    \en{value} yra objektas ir kiekviena operacija yra metodo kvietimas
    \cite[3]{scala-overview};
  \item ji yra funkcinė kalba ta prasme, kad kiekviena funkcija yra
    reikšmė \en{value}.
\end{itemize}
Toliau yra pristatoma kuo \plangname{Scala} pakeitė statinį, klasinį
objektinį programavimą, kuriam „atstovauja“ \plangname{Java}.
\plangname{Scala}, kaip funkcinės kalbos savybės šiame darbe nėra
nagrinėjamos.

\plangname{Scala} autoriai \cite{scalable-component-abstractions}
išskyrė tris programavimo kalbos elementus įgalinančius kurti įvairaus
dydžio \en{scalable} komponentus:
\begin{itemize}
  \item abstraktūs tipai – nariai \en{abstract type members};
  \item savo tipo anotavimas \en{selftype annotations};
  \item modulinė maišos kompozicija \en{modular mixin composition}.
\end{itemize}

\plangname{Scala}, be parametrizuojamųjų tipų \en{generics}, kuriuos
turi \plangname{Java}, turi dar vieną abstrakcijos mechanizmą:
\plangname{Scala} klasės gali turėti narius \en{member}, kurie
yra abstraktūs tipai. Abu mechanizmai yra pakeičiami vienas kitu:
\cite[10]{scala-overview} pateiktas pavyzdys, kaip parametrizuojamuosius
tipus galima būtų modeliuoti pasinaudojant abstrakčiais tipais, taip
pat nurodoma, kad yra įmanomas ir atvirkštinis variantas.
\plangname{Scala} kūrėjų \cite[11]{scala-overview} nurodytos
priežastys kalboje realizuoti abu mechanizmus yra skirtingi jų
panaudojimo atvejai: parametrizuojamus tipus yra siūloma naudoti kai
norime tiesiog galimybės klasės naudotojui nurodyti konkretų tipą
(tipinis pavyzdys būtų kolekcijos, \ref{lst:scala:abstract:members:4}
kodo fragmentas), o abstrakčius tipus tada, kai norime kliento kode
pasinaudoti abstrakčiu tipu. Pastaroji galimybė yra dažniausiai
naudojama su kita \plangname{Scala} „naujove“\footnote{
Visi trys programavimo kalbos elementai, kurie \plangname{Scala}
kūrėjų teigimu yra reikalingi komponentų kūrimui, (abstraktūs
tipai – nariai, savo tipo anotavimas ir modulinė maišos kompozicija)
egzistavo dar iki \plangname{Scala} sukūrimo, bet \plangname{Scala}
yra pirmoji programavimo kalba, kurioje realizuoti jie visi
\cite[2]{scalable-component-abstractions}.} – moduline maišos
kompozicija.

\begin{listing}[h]
  \begin{scalacode}
    import scala.collection.mutable.LinkedList
    val list = new LinkedList[String]
  \end{scalacode}
  \caption{Parametrizuotų tipų panaudojimo atvejis.}
  \label{lst:scala:abstract:members:4}
\end{listing}

Modulinė maišos kompozicija, tai klasių kūrimo mechanizmas
pasinaudojant fragmentų \en{trait} komponavimu. Fragmentą galimą
būtų apibrėžti kaip sąsają, kurios metodai gali turėti
realizacijas. Šiuo požiūriu fragmentai primena abstrakčias klases,
tik skirtingai nuo jų, fragmentų konstruktoriai negali turėti
parametrų. Fragmento konstrukcija leidžia pasinaudoti
multipaveldėjimo suteikiamais kodo perpanaudojimo privalumais
(klasę galima sukomponuoti iš kelių fragmentų), bet tuo pačiu, dėl
to, kad kompiliavimo metu paveldėjimo grafas yra ištiesinamas, neturi
taip vadinamos rombo problemos.

Klasių hierarchijos pakeitimas į fragmentų hierarchiją, net ir
nesinaudojant komponentinio galimybėmis, turi privalumą, kad
gauta sistema yra lengviau modifikuojama. Komponuojant klasę iš
fragmentų yra įmanoma „įlįsti“ į fragmentų hierarchijos vidų
ir ten atlikti pakeitimus, ko negalima padaryti su klasių hierarchija,
nes pastarosios ištiesinimas turi tenkinti savybę: „klasės hierarchijos
ištiesinimas visada turi tiesioginės tėvinės klasės hierarchijos
ištiesinimą, kaip galūnę“\cite[57p.]{scala-reference}. Galimybė,
atlikti pakeitimus viduje hierarchijos medžio, iliustruota programa,
pateikta \ref{lst:scala:mixin:1} kodo fragmente. Jos išvestis pateikta
\ref{lst:scala:mixin:2} fragmente. Šiame pavyzdyje klasė \scala{C2}
yra sukomponuojama iš fragmentų \scala{M3} ir \scala{T4} pritaikant
maišos kompoziciją.

\begin{listing}[H]
  \inputscala{e1/Demo}
  \caption{Fragmentų hierarchijos modifikavimas.}
  \label{lst:scala:mixin:1}
\end{listing}

\begin{listing}[H]
  \begin{textcode}
    List(C1, T4, T3, T2, T1)
    List(C2, T4, T3, M3, T2, T1)
  \end{textcode}
  \caption{\ref{lst:scala:mixin:1} kodo fragmente pateiktos programos
  išvestis.}
  \label{lst:scala:mixin:2}
\end{listing}

Turbūt svarbiausia maišos kompozicijos savybė yra ta, kad konkretus
narys visada užkloja abstraktų
\cite[6]{scalable-component-abstractions}. Ši savybė leidžia
sujungti du fragmentus, kurių nesieja bendra hierarchija. Tai yra
iliustruota programoje, pateiktoje \ref{lst:scala:mixin:3} kodo
fragmente. Jos išvestis yra pateikta \ref{lst:scala:mixin:4}
fragmente. Fragmento \scala{CachedCalculation} nariai
\scala{KeyType}, \scala{ValueType} ir \scala{calculate} yra abstraktūs
ir jie yra užklojami atitinkamų narių apibrėžtų fragmente
\scala{Factorial}. Fragmente \scala{Factorial} taip pat yra
abstraktus metodas \scala{lookup}, kurį užkloja konkretus metodas
\scala{lookup}, apibrėžtas \scala{CachedCalculation}.

\begin{listing}[H]
  \inputscala{e10/Demo}
  \caption{Maišos kompozicijos pavyzdys.}
  \label{lst:scala:mixin:3}
\end{listing}

\begin{listing}[H]
  \begin{textcode}
    Calculating 3
    Calculating 2
    Calculating 1
    3! = 6
    Calculating 5
    Calculating 4
    5! = 120
  \end{textcode}
  \caption{\ref{lst:scala:mixin:3} kodo fragmente pateiktos programos
  išvestis.}
  \label{lst:scala:mixin:4}
\end{listing}

Taikant maišos kompoziciją kartais prireikia, kad
\scala{this}\footnote{this yra specialus kintamasis rodantis į
objektą, kurio metodas dabar vykdomas.} tipas būtų ne tas fragmentas,
kurio metodas šiuo metu yra vykdomas, o kažkoks kitas, išvestinis
iš šio. Tai galima realizuoti vykdymo metu keičiant \scala{this}
tipą į norimą arba pasinaudoti \plangname{Scala} savojo tipo
anotacija. Pastarasis būdas turi privalumą, kad klaidos būtų randamos
kompiliavimo metu. Savojo tipo anotacijos panaudojimo pavyzdys pateiktas
\ref{lst:scala:selftype:1} kodo fragmente, o programos išvestis
– \ref{lst:scala:selftype:2}. Šiame pavyzdyje savojo tipo anotacija
yra \scala{this: Node =>} – su ja nurodoma, kad \scala{BaseNode}
viduje \scala{this} tipas yra \scala{Node}. Todėl kvietimas
\scala{show(this)} tampa galimas.

\begin{listing}[H]
  \inputscala{e11/Demo}
  \caption{Savojo tipo anotacijos panaudojimo pavyzdys.}
  \label{lst:scala:selftype:1}
\end{listing}

\begin{listing}[H]
  \begin{textcode}
    Root
    First child
    Second child
  \end{textcode}
  \caption{\ref{lst:scala:selftype:1} kodo fragmente pateiktos programos
  išvestis.}
  \label{lst:scala:selftype:2}
\end{listing}

Be savojo tipo anotacijos \ref{lst:scala:selftype:1} pavyzdyje buvo
panaudotas dar vienas svarbus programavimo kalbos elementas – vidinė
klasė (ir fragmentas). \cite[12]{scalable-component-abstractions}
nurodo, kad jie šio elemento neįtraukė tarp būtinų komponetiniam
programavimui, vien todėl, kad vidines klases palaiko pagrindinės
\en{mainstream} programavimo kalbos. Turbūt esminis skirtumas tarp
\plangname{Java} ir \plangname{Scala} vidinių klasių yra tai, kad
\plangname{Java} vidinės klasės tipas yra susietas su išorine klase,
o \plangname{Scala} su išoriniu objektu. \ref{lst:scala:selftype:1}
kodo fragmente pateiktame pavyzdyje kintamojo \scala{root} tipas
yra \scala{tree.SimpleNode}, kai \plangname{Java} jis būtų
\scala{Tree.SimpleNode}. Ši savybė yra naudinga, pavyzdžiui, tuo, kad 
leidžia statiniais metodais užtikrinti, kad skirtingų abstrakčios
gamyklos \en{abstract factory} realizacijų objektai nebūtų maišomi
tarpusavyje \cite[36]{scala-design-patterns}. Prireikus šią savybę
galima apeiti pasinaudojant tipų projekcija \en{type projection},
tai yra išreikštinai nurodant, kad vidinės klasės objekto tipas
turi priklausyti nuo išorinės klasės. \ref{lst:scala:selftype:1}
kodo fragmente nurodyta, kad kintamojo \scala{node} tipas yra
bet kokio \scala{Tree} tipo objekto \scala{SimpleNode} tipo objektas.

\section{Scala komponentinis modelis}

\cite[37]{heineman2001component} komponentinį modelį apibrėžė kaip
standartų rinkinį. (Žr.: \ref{section:component:model} skyrelį.)
Remiantis juo aprašytas \plangname{Scala} komponentinis modelis
pateiktas lentelėje (detalius aprašymus ką turėtų apibrėžti
kiekvienas iš standartų galima rasti
\cite[38-44]{heineman2001component}):

\xtable{
  w [ 2 | 5 ]
  a [ p | p ]
  h [ Standartas | Realizacija ]
  %
  e [
    realizacija |
    Komponentas gali būti realizuojamas, kaip @plangname{Scala} klasė
    arba kaip fragmentas.
    ]
  e [
    sąsajos |
    Komponentas servisus, kurių jam reikia, nurodo kaip abstrakčius
    narius, o kuriuos jis realizuoja – kaip konkrečius.
    ]
  e [
    įvardinimas |
    Komponentams globalūs vardai yra priskiriami taip pat, kaip ir
    @plangname{Java} klasėms – pagal hierarchines vardų sritis.
    ]
  e [
    meta duomenys |
    Nėra.
    ]
  e [
    tarpusavio sąveika |
    Kadangi komponentų egzemplioriai @en{component instances} yra
    objektai, tai jie sąveikauja keisdamiesi žinutėmis.
    ]
  e [
    pritaikymas |
    Komponentą pritaikyti konkrečiam atvejui galima jam pateikiant
    kitokias jo abstrakčių narių realizacijas (pavyzdžiui, kitą
    konkretų tipą). Taip pat komponento egzemplioriaus sukūrimo
    metu jį galima pritaikyti pasinaudojant konstruktoriaus parametrais
    bei parametrizuojamais tipais.
    ]
  e [
    kompozicija |
    Komponentai yra surišami kompiliavimo metu. Abstraktūs nariai
    su jų realizacijomis yra sujungiami pagal sutampančią
    @FIXME{Veicekausko vertimas: signatūrą}. Kompozicijos rezultatas
    yra naujas komponentas, kuris gali būti toliau komponuojamas.
    ]
  e [
    evoliucijos palaikymas |
    Kadangi komponentai yra surišami kompiliavimo metu, tai norint
    pakeisti vieną komponentą kitu reikia perkompiliuoti sistemą.
    Taip pat nėra galimybės turėti dvi to paties komponento versijas
    vienoje sistemoje.
    ]
  e [
    supakavimas ir įdiegimas |
    Nėra.
    ]
}

Pagal \cite[40]{heineman2001component} meta duomenys, tai yra
informacija apie sąsajas, komponentus ir jų ryšius. Ji leidžia
dinamiškai sujungti komponentus, taip pat ir nutolusius. Kadangi
\plangname{Scala} komponentai yra sujungiami statiškai kompiliavimo
metu, tai ši informacija nėra reikalinga.

Viena iš esminių komponento savybių yra tai, kad jis yra fizinis
diegimo vienetas. Tačiau \plangname{Scala} komponentinis modelis
neapibrėžia kokiu būdu turėtų būti platinami ir įdiegiami
komponentai. Be to komponentų naudotojas tam, kad galėtų juos
sukomponuoti į sistemą, turi turėti jų išeities tekstus. Taigi
galima būtų teigti, kad \plangname{Scala} komponentinis modelis šios
apibrėžimo dalies netenkina.

Tam, kad \plangname{Scala} komponentinis modelis visiškai tenkintų
komponentinio modelio apibrėžimą, reikia apibrėžti komponentų
platinimo būdus. Kadangi programų kodas dažniausiai yra laikomas
versijų kontrolės sistemose, tai kaip vienas iš būdų platinti
\plangname{Scala} komponentus, būtų pasinaudojant jomis. Pavyzdžiui,
\progname{Git}\footnote{\url{http://git-scm.com/}} ir
\progname{Mercurial}\footnote{\url{http://mercurial.selenic.com/}} palaiko
įdėtines saugyklas, taigi komponentų, iš kurių sudaryta sistema,
versijų kontrolės saugyklas galima būtų nurodyti, kaip įdėtines
kuriamos sistemos versijų kontrolės sistemai. Taigi tokiu
būdu papildžius \plangname{Scala} komponentinį modelį gautume
komponentinį modelį atitinkantį \cite[37]{heineman2001component}
pateiktą apibrėžimą.

\section{\plangname{Scala} komponentinio modelio savybės}

Šiame skyrelyje pabandyta palyginti \plangname{Scala} komponentinį
modelį su kitais komponentiniais modeliais, remiantis
\cite{classification-framework-for-scm} klasifikacija. Klasifikacijos
autoriai išskyrė tokias komponentinių modelių klasifikavimo dimensijas:
\begin{description}
  \item[gyvavimo ciklas] – kiek komponentinis modelis paremia kiekvieną
    iš komponento gyvavimo ciklo stadijų;
  \item[konstravimas] – kokiu būdu yra sukonstruojama sistema iš
    komponentų;
  \item[ekstra-funkcinės savybės] – ką komponentinis modelis siūlo
    ekstra-funkcinių savybių specifikavimui, valdymui ir kompozicijai.
\end{description}

\cite[596]{classification-framework-for-scm} gyvavimo ciklą suskirstė į
keturias stadijas. Komentarai, kaip \plangname{Scala} komponentinis
modelis palaiko kiekvieną iš jų, pateikti lentelėje:
\xtable{
  w [ 2 | 5 ]
  a [ p | p ]
  h [ Stadija | Jos palaikymas ]
  %
  e [
    modeliavimas |
    Specialių metodų ar įrankių nėra. Iš dalies (be konstrukcijų
    perimtų iš funkcinių programavimo kalbų) galima modeliuoti
    pasinaudojant @progname{UML} (Unified Modeling Language), ją
    papildžius tokiomis konstrukcijomis kaip fragmentas ir
    modulinė maišos kompozicija. Galima realizacija pateikiama
    @cite&L145&R{rachimow2009scala}.
    ]
  e [
    realizacija |
    Galima teigti, kad @plangname{Scala} komponentinis modelis yra
    labiausiai orientuotas į šią komponento gyvavimo ciklo dalį.
    Komponentų realizavimui jis pateikia ne tik taisykles, bet
    ir kalbos konstrukcijas. (Skirtingai, pavyzdžiui, nuo @progname{EJB} ar
    @progname{OSGi}, kurie tik apriboja programavimo su @plangname{Java}
    būdus.) Taip pat, kitaip nei kai kurių kitų komponentinių modelių,
    @plangname{Scala} komponentinio modelio atveju, šios stadijos
    rezultatas yra komponento išeities tekstai, o ne sukompiliuotas
    komponentas.
    ]
  e [
    pakavimas |
    Šios stadijos @plangname{Scala} komponentinis modelis nepalaiko.
    ]
  e [
    įdiegimas |
    Komponentai yra surišami kompiliavimo metu.
    ]
}

\cite{classification-framework-for-scm} konstravimo dimensiją suskirstė
į tris dalis:
\begin{enumerate}
  \item sąsajų specifikavimas – komponentų sujungimo taškų nurodymas;
  \item surišimas – ryšių tarp komponentų sukūrimas;
  \item sąveika – komponentų tarpusavio komunikacija.
\end{enumerate}

Nagrinėjant \plangname{Scala} komponentinio modelio sąsajų
specifikaciją pagal \cite[599]{classification-framework-for-scm}
pateiktus kriterijus galima būtų išskirti:
\begin{enumerate}
  \item sąsajos yra paremtos operacijomis (o ne prievadais);
  \item yra aiškiai atskiriamos reikalaujamos \en{required interface}
    ir teikiamos sąsajos \en{provided interface};
  \item sąsajos yra aprašomos \plangname{Scala} programavimo kalba;
  \item sąsajų suderinamumas yra tikrinamas tik sintaksiniu
    požiūriu.
\end{enumerate}
Nagrinėjant surišimo mechanizmus galima būtų išskirti:
\begin{enumerate}
  \item komponentus galima sujungti tik tiesiogiai, tai yra sujungimai
    \en{connectors}, kaip atskiri elementai nėra išskiriami;
  \item kadangi kompozicijos rezultatas yra komponentas, tai
    galime teigti, kad \plangname{Scala} palaiko pilną vertikalią
    kompoziciją: delegavimo tipo jungimą (ansamblio prašomos /
    teikiamos sąsajos sujungiamos su vidinių jo komponentų
    prašomomis / teikiamomis sąsajomis), agregavimo jungimą
    (visos vidinių komponentų sąsajos yra pasiekiamos per ansamblio
    sąsajas) ir naujojo komponento ekstra-funkcinės savybės
    tenkina komponentinio modelio reikalavimus, kadangi visi
    \plangname{Scala} fragmentai ir klasės gali būti naudojami,
    kaip komponentai.
\end{enumerate}
Nagrinėjant komponentų sąveiką galima būtų išskirti:
\begin{enumerate}
  \item sąveikos stilius tarp komponentų yra užklausa-atsakymas
    \en{request-response};
  \item kadangi komunikacija tarp komponentų vyksta kviečiant metodus,
    tai ji yra sinchroninio tipo.
\end{enumerate}

\cite[602]{classification-framework-for-scm} išskyrė tokius
aspektus, pagal kuriuos galima nagrinėti kaip komponentinis
modelis palaiko ekstra-funkcines savybes \en{extra-functional
properties}:
\begin{enumerate}
  \item ekstra-funkcinių savybių valdymas;
  \item ekstra-funkcinių savybių specifikavimas;
  \item ekstra-funkcinių savybių komponavimas.
\end{enumerate}
\plangname{Scala} komponentinis modelis nei vieno iš šių aspektų
išreikštinai nepalaiko.

Apibendrinant galima būtų paminėti, kad \plangname{Scala} komponentinis
modelis iš kitų modelių, nagrinėtų
\cite{classification-framework-for-scm}, išsiskiria tuo, kad
pilnai palaiko vertikalią kompoziciją. Iš to atsiranda privalumų
ir trūkumų. Kadangi komponentai, iš kurių yra surinktas ansamblis
yra paslepiami jo viduje, tai toks modelis labiau tinkamas realaus
pasaulio modeliavimui, nei modelis palaikantis tik horizontalų jungimą.
Pavyzdžiui, modeliuojant su \plangname{Scala} komponentas variklis po
kompozicijos būtų paslėptas komponente automobilis ir galima būtų
naudotis automobiliu nieko nežinant apie jo variklį. Tuo tarpu
modeliuojant su komponentiniu modeliu, kuris palaiko tik horizontalų
jungimą, komponentai variklis ir automobilis būtų atskirai, todėl
kiekvieną kartą kažką darant su automobiliu, reikėtų nepamiršti
ir jo variklio. Kitaip tariant \plangname{Scala} komponentais, kurie
yra ansambliai, yra žymiai paprasčiau naudotis, nei kitų modelių
ansambliais, kurie yra tiesiog komponentų rinkiniai. Ši savybė taip
pat turi ir trūkumą: kadangi komponentai yra paslepiami viduje, tai
norint juos pakeisti gali reikėti išardyti ansamblį.

\FIXME{Reikėtų paminėti, kad komponentai gali būti vidiniai
\en{nested}, todėl ir gaunasi, kad jie iš išorės nematomi, kai juos
gaubiantys komponentai, kurie turi metodus jų kūrimui, yra matomi.}

Taigi galima teigti, kad \plangname{Scala} kūrėjams pavyko sukurti
komponentinį modelį, kuris leistų taip pat lengvai dirbti ir
su labai mažais komponentais (iki kelių dešimčių eilučių) ir su
dideliais (visa sistema, ar posistemė, pavyzdžiui \plangname{Scala}
kompiliatorius).

\section{Scala komponentinis prieš Java objektinį}

TODO: Paprastas objektinis, ar objektinis su dizaino šablonais? Jei su
dizaino šablonais, tai pranašumo nebus, nes jie ir yra tam, kad
leistų užtikrinti modifikuojamumą ir perpanaudojamumą užtikrindami
SOLID. Turbūt netgi išeitų parodyti, kad sistemą, kurioje yra
tenkinamas SOLID, galima lengvai išskaidyti į komponentus.
Kitaip tariant vienintelis \plangname{Scala} pranašumas prieš
\plangname{Java} būtų tai, kad ji komponentinio konstrukcijas
palaiko ne programavimo disciplina, o kalbos priemonėmis.

\Chapter*{Rezultatai ir išvados}

% Rezultatų ir išvadų dalyje turi būti aiškiai išdėstomi
% pagrindiniai darbo rezultatai (kažkas išanalizuota, kažkas sukurta,
% kažkas įdiegta) ir pateikiamos išvados (daromi nagrinėtų problemų
% sprendimo metodų palyginimai, teikiamos rekomendacijos, akcentuojamos
% naujovės).

Šiame darbe buvo pristatyta, kas yra komponentinis modelis, išskirtos
komponento savybės bei parodyta, kad \plangname{Scala} programavimo
kalba ir \progname{Debian} \gls{package-management-system}{paketų
tvarkymo sistema} yra komponentinės technologijos.

Nors objektinis programavimas ir komponentinis sistemų kūrimo būdas
yra du skirtingi programų sistemų kūrimo lygiai, bet
\plangname{Scala} kūrėjams pavyko šiuos lygius sujungti į vieną,
sukuriant kalbą, kuria yra įmanoma programuoti komponentiškai.
\plangname{Scala} komponentinis modelis, kaip ir kiti
\cite{classification-framework-for-scm} analizuoti komponentiniai
modeliai, akcentuoja praplečiamumą ir modfikuojamumą, o tai, kad
nėra apibrėžtas \plangname{Scala} komponentų perdavimo formatas,
apsunkina jos kodo perpanaudojamumą. Kaip pagrindinį \plangname{Scala}
komponentinio modelio privalumą lyginant su kitais komponentiniais
modeliais galima būtų išskirti \plangname{Scala} naudojimo paprastumą.

Iš \progname{Debian} paketų tvarkymo sistemos ir egzistuojančių
komponentinių modelių analizės galima būtų teigti, kad
perpanaudojamumą pagerina komponentų saugyklų egzistavimas bei galimybė
diegiant komponentą automatiškai gauti visus jo funkcionavimui
reikiamus komponentus. Taip pat remiantis \progname{Debian} paketų
tvarkymo sistemos analize galima būtų kelti hipotezę, kad
sistemos modifikuojamumą padidina komponentų priklausymas tik nuo
abstrakcijų, o ne nuo konkrečių kitų komponentų.

Apibendrinant, galima būtų kelti hipotezę, kad norint, jog
komponentinis modelis pasižymėtų perpanaudojamumu, modifikuojamumu ir
praplečiamumu, reikia, kad jo komponentai be išimčių turėtų visas
\ref{section:component:descriptions} skyrelyje nurodytas komponento
savybes ir komponentinis modelis papildomai apibrėžtų komponentų
platinimo bei jų automatinio įdiegimo kartu su priklausomybėmis
mechanizmus.

\Bibliography
\Chapter*{Sąvokų apibrėžimai}

\begin{glossary}

  \begin{entry}%
    {evolutionary-software-development-process}%
    {evoliucinis programinės įrangos kūrimo procesas}%
    [evolutionary software development process]

    Procesas, kurio metu programinė įranga papildoma nauju funkcionalumu,
    neliečiant jau egzistuojančio.
    
  \end{entry}

  \begin{entry}%
    {component-oriented-software-development}%
    {komponentinis programinės įrangos kūrimas}%
    [component oriented software development]

    Programinės įrangos kūrimo būdas, kai ji yra surenkama iš atskirų,
    išreikštinai nesusijusių gabalų, kurie yra vadinami komponentais.
    
  \end{entry}

  \begin{entry}%
    {trait}%
    {fragmentas}%
    [trait]

    Programavimo kalbos \gls{scala}{Scala} konstrukcija. Scala
    fragmentai nuo Java sąsajų skiriasi tuo, kad gali turėti metodų
    realizacijas. Taigi tuo jie primena abstrakčias klases, bet nuo
    pastarųjų skiriasi tuo, kad jų konstruktoriai negali turėti
    parametrų.
  
  \end{entry}

  \begin{entry}%
    {scala}%
    {Scala}%
    []%
    [\url{http://www.scala-lang.org}]

    Multiparadigminė programavimo kalba.
    
  \end{entry}

  \begin{entry}%
    {abstract-type-member}%
    {abstraktus tipas-atributas}%
    [abstract type member]

    Abstraktūs tipai, kaip atributai yra objektiškai orientuota
    \gls{scala}{Scala} konstrukcija, kuri leidžia abstrahuotis nuo
    komponento funkcionavimui reikalingų komponentų. Plačiau
    jie aprašyti \cite[8]{scala-overview}.
    
  \end{entry}

  \begin{entry}%
    {selftype-annotation}%
    {savo tipo anotacija}%
    [selftype annotation]

    \gls{scala}{Scala} konstrukcija leidžianti specialiajam kintamajam
    \varname{this} priskirti kitą tipą.

  \end{entry}

  \begin{entry}%
    {modular-mixin-composition}%
    {modulinė maišos kompozicija}%
    [modular mixin composition]

    Komponentų jungimo būdas realizuotas programavimo kalboje
    \gls{scala}{Scala}.
    
  \end{entry}

  \begin{entry}%
    {statically-typed-programming-language}%
    {statiškai tipizuota programavimo kalba}%
    [statically typed programming language]%
    [\url{http://en.wikipedia.org/wiki/Static_typing\#Static_typing}]

    Programavimo kalba, kurioje tipų patikrinimas yra atliekamas
    kompiliavimo metu.\cite[2]{Madsen:1990:STO:97946.97964}
    
  \end{entry}

  \begin{entry}%
    {dynamically-typed-programming-language}%
    {dinamiškai tipizuota programavimo kalba}%
    [dynamically typed programming language]%
    [\url{http://en.wikipedia.org/wiki/Dynamic_typing\#Dynamic_typing}]

    Programavimo kalba, kurioje tipų patikrinimas yra atliekamas vykdymo
    metu \cite[2]{Madsen:1990:STO:97946.97964}. Kalbose, kurios naudoja
    dinaminį tipų tikrinimą, tipus turi turi reikšmės, o ne kintamieji.
    
  \end{entry}

  \begin{entry}%
    {class-based-programming-language}%
    {klasinė programavimo kalba}%
    [class based programming language]%
    [\url{http://en.wikipedia.org/wiki/Class-based_programming}]

    \gls{object-oriented-programming-language}{Objektinė programavimo
    kalba}, kurioje objektai yra apibrėžiami,
    kaip klasių egzemplioriai. \TODO{Pavedimas} yra pasiekiamas
    pasinaudojant klasių paveldėjimo konstrukciją.

  \end{entry}%
    
  \begin{entry}%
    {prototype-based-programming-language}%
    {prototipinė programavimo kalba}%
    [prototype based programming language]%
    [\url{http://en.wikipedia.org/wiki/Prototype-based_programming}]

    \gls{object-oriented-programming-language}{Objektinė programavimo
    kalba}, kurioje objektai yra apibrėžiami, nukopijuojant jau
    egzistuojančių objektų, kurie yra vadinami prototipais,
    apibrėžimus arba nurodant, kad objektui neradus, kaip jis
    turėtų apdoroti gautąją žinutę, ji turėtų būti persiųsta
    jo prototipui.\cite[176]{Wegner:1987:DOL:38807.38823}
    
  \end{entry}
    
  \begin{entry}%
    {object-oriented-programming-language}%
    {objektiškai orientuota programavimo kalba}%
    [object oriented programming language]%
    [\url{http://en.wikipedia.org/wiki/Object_oriented}]

    Programavimo kalba, su kuria galima programuoti pasinaudojant
    objekto abstrakcija bei kurioje yra realizuotas 
    \gls{object-description-sharing-mechanism}{objektų apibrėžimų
    dalinimosi mechanizmas}.
    
  \end{entry}

  \begin{entry}%
    {object-description-sharing-mechanism}%
    {objekto aprašymo dalinimosi mechanizmas}%
    [object description sharing mechanism]

    Mechanizmas leidžiantis egzistuojančių objektų apibrėžimus panaudoti
    apibrėžiant naujus objektus.
    \gls{object-oriented-programming-language}{Objektinėse programavimo
    kalbose} tai gali būti realizuota, kaip \gls{delegation}{delegavimas}
    arba \gls{class-inheritance}{klasių paveldėjimas}.
    
  \end{entry}

  \begin{entry}%
    {delegation}%
    {delegavimas}%
    [delegation]

    \gls{object-description-sharing-mechanism}{Objektų apibrėžimų
    dalinimosi mechanizmas}, kai naujas objektas yra apibrėžiamas
    pasinaudojant jau egzistuojančio objekto apibrėžimu. Šis
    mechanizmas vadinamas delegavimu, nes neradus žinutę atitinkančio
    metodo ji yra persiunčiama (deleguojama) tėviniam objektui.
    
  \end{entry}

  \begin{entry}%
    {class-inheritance}%
    {klasių paveldėjimas}%
    [class inheritance]

    \gls{object-description-sharing-mechanism}{Objektų apibrėžimų
    dalinimosi mechanizmas}, kai kiekvienas objektas yra laikomas
    kokios nors klasės egzemplioriumi (tai yra, jis apibrėžiamas
    pasinaudojant klase), o klasės gali paveldėti savybes vienos
    iš kitų.
    
  \end{entry}

  \begin{entry}%
    {strongly-typed-programming-language}%
    {stipriai tipizuota programavimo kalba}%
    [strongly typed programming language]%
    [\url{http://en.wikipedia.org/wiki/Strong_typing}]

    Tipizacijos stiprumas yra tai, kokį kiekį informacijos turi
    konkrečios išraiškos tipas. Idealiu atveju stipriai tipizuotoje
    kalboje iš nuorodos į objektą tipo visada galima pasakyti kokias
    žinutes galima siųsti tam
    objektui.\cite[1]{Madsen:1990:STO:97946.97964}
    
  \end{entry}

  \begin{entry}%
    {weakly-typed-programming-language}%
    {silpnai tipizuota programavimo kalba}%
    [weakly typed programming language]%
    [\url{http://en.wikipedia.org/wiki/Strong_typing}]

    Tipizacijos stiprumas yra tai, kokį kiekį informacijos turi
    konkrečios išraiškos tipas. Silpnai tipizuotoje kalboje neįmanoma
    nustatyti kokias žinutes galima siųsti objektui iš nuorodos į jį
    tipo.\cite[1]{Madsen:1990:STO:97946.97964}
    
  \end{entry}

  \begin{entry}%
    {dynamic-dispatch}%
    {dinaminis susiejimas}%
    [dynamic dispatch, dynamic binding]%
    [\url{http://en.wikipedia.org/wiki/Dynamic_dispatch}]

    Nusprendimo, kokią procedūrą vykdyti gavus žinutę, procesas,
    atliekamas programos vykdymo metu.
    \cite[225]{types-and-programming-languages}

  \end{entry}

  \begin{entry}%
    {encapsulation}%
    {uždarumas}%
    [encapsulation]

    Savybė, kai vidinė objekto struktūra yra slepiama nuo išorės
    naudotojų. \cite[225]{types-and-programming-languages}
    
  \end{entry}

  \begin{entry}%
    {subtyping}%
    {potipiai}%
    [subtyping]

    \gls{object-oriented-programming-language}{Objektinės programavimo
    kalbos} savybė, kai norint pasinaudoti objektu mums rūpi tik
    jo sąsaja ir mes galime naudoti objektą $I$ vietoj
    objekto $J$, jei objekto $J$ sąsaja yra objekto $I$ sąsajos poaibis.
    \cite[226]{types-and-programming-languages}
    
  \end{entry}

  \begin{entry}%
    {open-recursion}%
    {atvira rekursija}%
    {open recursion}

    Specialaus kintamojo (dažniausiai jis vadinamas „this“, arba
    „self“) egzistavimas, kuriuo pasinaudojant galima kreiptis į
    kitus to paties objekto metodus.
    \cite[226]{types-and-programming-languages}
    
  \end{entry}

  \begin{entry}%
    {component-instance}%
    {komponento egzempliorius}%
    {component instance}

    Komponento inicializacijos rezultatas.
    
  \end{entry}

  % Nesutvarkyti.

  \begin{entry}%
    {polymorphism}%
    {polimorfismas}%
    [polymorphism]%
    [%
    \url{http://en.wikipedia.org/wiki/Polymorphism_(computer_science)}, %
    \url{http://en.wikipedia.org/wiki/Polymorphism_in_object-oriented_programming}%
    ]%

    Programavimo kalbos savybė, kuri leidžia su skirtingų tipų duomenimis
    dirbti naudojantis ta pačia sąsaja. Išskiriami trys polimorfizmo tipai:
    \begin{itemize}
      \item \emph{Ad-hoc} polimorfizmas \en{Ad-hoc polymorphism} –
        iš esmės, tai yra funkcijų perdengimas;
      \item parametrinis polimorfizmas \en{parametric polymorphism} –
        programuojama taip, kad kodas nepriklausytų nuo duomenų tipo
        (pavyzdys būtų C++ šablonai ir Java generics);
      \item potipio polimorfizmas \en{subtype polymorphism} – 
        jei $T$ yra $S$ potipis, tai $T$ galima naudoti vietoje $S$.
    \end{itemize}
    
  \end{entry}
\end{glossary}

\begin{description}

  \item[metodo užklotis \en{method overriding}]
    Objektinės programavimo kalbos savybė, kuri leidžia paveldinčiai klasei
    realizuoti metodą, kurį jau yra realizavusi kažkuri iš jos tėvinių
    klasių. Paveldinčios klasės metodo realizacija užkloja (paslepia)
    tėvinės klasės metodą.

  \item[funkcijos perdengimas \en{function overloading}]
    Programavimo kalbos savybė, kuri leidžia aprašyti kelias funkcijas
    turinčias tą patį vardą, kurios yra atskiriamos pagal jų argumentų
    tipus.

  \item[įdėtinė klasė \en{nested class, static inner class}]
    Java kalboje, tai klasė apibrėžta kitos klasės viduje, bet
    kurios objekto sukūrimui nėra būtinas gaubiančiosios
    klasės objektas. Ji apibrėžiama su \verb|static| konstrukcija.
    Scala neturi įdėtinių klasių.

  \item[vidinė klasė \en{inner class}]
    Java kalboje, tai klasė apibrėžta kitos klasės viduje. Jos
    objekto sukūrimui yra būtinas gaubiančiosios klasės objektas.
    Scala kalboje visos klasės apibrėžtos kitų klasių viduje
    yra vidinės klasės.

\end{description}

\appendix
\Chapter*{Priedai}
\def\thesection{\arabic{section} priedas.}
\section[\hspace{1.5em} Kodo pavyzdžiai]{Kodo pavyzdžiai}

\begin{scalainterpreterlisting}
  \inputscalai{AbstractMembers4}
  \ucaption{Parametrizuotų tipų panaudojimo atvejis}
  \label{lst:scala:abstract:members:4}
\end{scalainterpreterlisting}

\begin{scalacodelisting}
  \inputscala[lastline=12]{e14/Demo}
  \ucaption{Abstrakčių tipų – narių panaudojimas}
  \label{lst:scala:abstract:members:5}
\end{scalacodelisting}

\begin{scalainterpreterlisting}
  \inputscalai{AbstractMembers6}
  \ucaption{\ref{lst:scala:abstract:members:5} kodo pavyzdyje pateiktų
  klasių panaudojimas}
  \label{lst:scala:abstract:members:6}
\end{scalainterpreterlisting}

\begin{scalacodelisting}
  \inputscala[lastline=22]{e1/Demo}
  \ucaption{Fragmentų hierarchijos modifikavimas}
  \label{lst:scala:mixin:1}
\end{scalacodelisting}

\begin{scalainterpreterlisting}
  \inputscalai{Mixin2}
  \ucaption{\ref{lst:scala:mixin:1} kodo pavyzdyje pateiktos hierarchijos
  panaudojimas}
  \label{lst:scala:mixin:2}
\end{scalainterpreterlisting}

\begin{scalacodelisting}
  \inputscala[lastline=32]{e10/Demo}
  \ucaption{Maišos kompozicijos panaudojimas}
  \label{lst:scala:mixin:3}
\end{scalacodelisting}

\begin{scalainterpreterlisting}
  \inputscalai{Mixin4}
  \ucaption{\ref{lst:scala:mixin:3} kodo pavyzdyje sukomponuotos
  klasės \scala{CachedFactorial} panaudojimas}
  \label{lst:scala:mixin:4}
\end{scalainterpreterlisting}

\begin{scalacodelisting}
  \inputscala[lastline=19]{e11/Demo}
  \ucaption{Savojo tipo anotacijos panaudojimas}
  \label{lst:scala:selftype:1}
\end{scalacodelisting}

\begin{scalainterpreterlisting}
  \inputscalai{Selftype2}
  \ucaption{\ref{lst:scala:selftype:1} kodo pavyzdyje apibrėžtų
  klasių panaudojimas}
  \label{lst:scala:selftype:2}
\end{scalainterpreterlisting}

\begin{scalainterpreterlisting}
  \inputscalai{Compiler1}
  \ucaption{Scala kompiliatoriaus objekto sukūrimas}
  \label{lst:scala:compiler:1}
\end{scalainterpreterlisting}


\end{document}
