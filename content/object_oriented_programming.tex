\chapter{Objektinės paradigmos apibrėžimas}

\section{Įvairūs objektinės programavimo kalbos apibrėžimai}

\subsection{„The Treaty of Orlando“}

1987 metais ACM OOPSLA paskelbė dokumentą, pavadintą „The Treaty of
Orlando“ \cite{Lieberman:1987:TO:62139.62144}, kuriame nurodoma, kad
objektinių programavimo kalbų pagrindinis bruožas yra galimybė objektui
pačiam spręsti ką jis daro, gavęs žinutę. (Tai iš esmės skiriasi nuo
modulinio programavimo, kur visada yra įvykdoma ta pati procedūra.)
Dokumente yra minimi du būdai, kaip tai galima pasiekti:
\begin{itemize}
  \item žinučių persiuntimas \en{messages forwarding} ir
  \item dinaminis paveldėjimas \en{dynamic inheritance} (pavyzdžiui,
    \cite[272]{cpp-design-evolution} aprašomas bandymas C++
    kalboje realizuoti galimybę konstruktoriui perduoti rodyklę, kuri
    turėtų būti naudojama, kaip bazinė klasė).
\end{itemize}

\subsection{Benjamin C. Pierce}

2002 metais Benjamin C. Pierce savo knygoje „Types and programming
languages“\cite[225-227]{types-and-programming-languages} išvardino
esmines savybes, kurias turi dauguma objektinių programavimo kalbų: 
\begin{description}
  \item[dinaminis susiejimas] \en{dynamic dispatch, dynamic binding} –
    objekto reakcija į gautą žinutę nustatoma vykdymo metu;
  \item[uždarumas, inkapsuliacija] \en{encapsulation} – vidinė objekto
    struktūra yra slepiama;
  \item[potipiai] \en{subtyping} – kai norime pasinaudoti objektu, 
    tai mums rūpi tik jo sąsaja ir mes galime naudoti objektą $I$ vietoj
    objekto $J$, jei objekto $J$ sąsaja yra objekto $I$ sąsajos poaibis;
  \item[paveldėjimas, pavedimas] \en{inheritance, delegation} – galimybė
    perpanaudoti jau egzistuojantį kodą: tai gali būti pasiekiama
    objektų kūrimui naudojant klases, kurios gali paveldėti kai kurias
    savybes iš tėvinių klasių, arba naudojant žinučių persiuntimą;
  \item[atvira rekursija] \en{open recursion} – specialaus kintamojo
    (dažniausiai jis vadinamas „this“, arba „self“) egzistavimas, kuriuo
    pasinaudojant galima kreiptis į kitus to paties objekto metodus.
    Svarbi savybė yra \en{late-bound}.
\end{description}

\subsection{John Mitchell}

Pagal John Mitchell \cite[277]{concepts-in-programming-languages}
objektinė programavimo kalba \en{object-oriented language} privalo
turėti:
\begin{description}
  \item[objektus] – „paslėpti“ duomenys ir operacijos, kurias galima
    atlikti su jais;
  \item[dinaminis susiejimas] \en{dynamic lookup} – objektas pats
    sprendžia, kaip sureaguoti į žinutę ir skirtingi objektai gali
    į tą pačią žinutę sureaguoti skirtingai;
  \item[abstrakcija] \en{abstraction} – realizacijos detalės yra
    paslėptos ir veiksmus galima atlikti tik per sąsają (paprastai
    objektų sąsaja yra aibė viešų metodų, kurie atlieka veiksmus
    su paslėptais duomenimis);
  \item[potipiai] \en{subtyping} – jei objektas a turi visą funkcionalumą,
    kurį turi objektas b, tai objektą a galima naudoti visur vietoj
    objekto b;
  \item[paveldėjimas] \en{inheritance} – galimybė perpanaudoti vieno
    objekto apibrėžimą, apibrėžiant kitą objektą.
\end{description}

\subsection{Grady Booch}

Knygoje \cite[41-42]{Booch:2007:OAD:1407387} teigiama, kad objektiškai
orientuotas programavimas yra programų kūrimo metodika, kai programos
yra sudarytos iš tarpusavyje komunikuojančių objektų grupių; kiekvienas
objektas yra priskirtas kuriai nors klasei ir visos klasės yra
tarpusavyje susietos paveldėjimo ryšiu. (Nurodomas senesnis
apibrėžimas (1985 metų) iš esmės atmetant „The Treaty of
Orlando“.)

\subsection{Michael L. Scott}

Michael L. Scott \cite[529-530]{programming-language-pragmatics}
išskyrė tik šias savybes:
\begin{itemize}
  \item uždarumas, inkapsuliacija;
  \item paveldėjimas;
  \item dinaminis susiejimas.
\end{itemize}

\subsection{Peter Wegner}

TODO: Išanalizuoti.\cite{Wegner:1987:DOL:38807.38823}
Failas: p168-wegner.pdf

\section{Klasinis objektiškai orientuotas programavimas}

Kadangi objektinio programavimo kalbos, kuriose objektų kūrimas yra
paremtas klasėmis, yra žymiai populiaresnės (FIXME: Pagrįsti.), tai
toliau objektiniu programavimu bus vadinamas būtent objektinis
programavimas panaudojant klases \en{class-based programming}. Jo
pagrindinės savybės būtų:
\begin{description}
  \item[dinaminis susiejimas] – objektas pats sprendžia, kaip sureaguoti
    į žinutę ir skirtingi objektai gali į tą pačią žinutę sureaguoti
    skirtingai;
  \item[uždarumas, inkapsuliacija] – vidinė objekto struktūra yra
    slepiama;
  \item[potipiai] – kai norima pasinaudoti objektu, tai mums rūpi tik
    jo sąsaja ir mes galime naudoti objektą $a$ vietoj objekto $b$,
    jei objekto $b$ sąsaja yra objekto $a$ sąsajos poaibis;
  \item[paveldėjimas] – galimybė perpanaudoti vieno objekto apibrėžimą,
    apibrėžiant kitą objektą.
\end{description}

\section{Sutvarkyti}

\begin{defn}[Objektas]
  Esybė sudaryta iš trumpalaikio atributų ir veiksmų junginio.

  Objekto savybės:
  \begin{description}
    \item[tapatybė] – savybė, kuri leidžia atskirti objektą nuo kitų
      objektų;
    \item[būsena] – objekto saugomi duomenys;
    \item[elgsena] – veiksmai, kuriuos galima atlikti su objektu.
  \end{description}

  Pagal \url{http://en.wikipedia.org/wiki/Object_(computer_science)}.

  TODO: Aprašyti objektą pagal \cite[38]{Booch:2007:OAD:1407387}.

  TODO: Aprašyti objektą ir objektinę paradigmą pagal
  \cite[37]{cs-beyond-object-oriented-programming}.

\end{defn}

TODO: Pridėti paaiškinimą, kodėl man rūpi tik pirmas:
\begin{itemize}
  \item \url{http://en.wikipedia.org/wiki/Class-based_programming},
  \item \url{http://en.wikipedia.org/wiki/Prototype-based_programming}.
\end{itemize}

TODO: Nurodyti, kodėl miniu būtent šituos šaltinius. (Kodėl būtent jie
yra svarbūs.)

TODO: Pasinagrinėti \cite[38]{Booch:2007:OAD:1407387} pateikiamus
šaltinius.

TODO: Užmesti akį į:
\url{http://www.ipipan.gda.pl/~marek/objects/TOA/OOMethod/mcr.html}.

TODO: Penki gero objektinio programavimo stiliaus principai:
\url{http://en.wikipedia.org/wiki/Solid_(object-oriented_design)}

\subsection{Privalomos savybės}
\subsection{Papildomos savybės}

\begin{description}
  \item[objektų klasės] \en{classes of objects};
  \item[FIXME: klasių objektai] \en{instances of classes};
  \item[metodai susieti su objektais] \en{methods which act on the
    attached objects};
  \item[žinučių siuntimas (kitiems procesams)] \en{message passing}
    TODO: \url{http://en.wikipedia.org/wiki/Message_passing};
  \item[abstrakcija] \en{abstraction}
    TODO: \url{http://en.wikipedia.org/wiki/Abstraction_(computer_science)}
\end{description}
