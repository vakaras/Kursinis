\chapter{Objektinės paradigmos apibrėžimas}

\section{Privalomos savybės}

1987 metais ACM OOPSLA paskelbė dokumentą, pavadintą „The Treaty of
Orlando“ \cite{Lieberman:1987:TO:62139.62144}, kuriame nurodoma, kad
objektinių programavimo kalbų pagrindinis bruožas yra galimybė objektui
pačiam spręsti ką jis daro, gavęs žinutę. (Tai iš esmės skiriasi nuo
modulinio programavimo, kur visada yra įvykdoma ta pati procedūra.)
Dokumente yra minimi du būdai, kaip tai galima pasiekti:
\begin{itemize}
  \item žinučių persiuntimas \en{messages forwarding} ir
  \item dinaminis paveldėjimas \en{dynamic inheritance} (TODO:
    Išsiaiškinti).
\end{itemize}

2002 metais Benjamin C. Pierce savo knygoje „Types and programming
languages“\cite[225-227]{types-and-programming-languages} išvardino
esmines savybes, kurias turi dauguma objektinių programavimo kalbų: 
\begin{description}
  \item[dinaminis susiejimas] \en{dynamic dispatch, dynamic binding} –
    objekto reakcija į gautą žinutę nustatoma vykdymo metu;
    TODO: \url{http://en.wikipedia.org/wiki/Dynamic_dispatch}
  \item[uždarumas, inkapsuliacija] \en{encapsulation} – vidinė objekto
    struktūra yra slepiama;
  \item[FIXME: subtyping] \en{subtyping} – kai norime pasinaudoti objektu, 
    tai mums rūpi tik jo sąsaja ir mes galime naudoti objektą $I$ vietoj
    objekto $J$, jei objekto $J$ sąsaja yra objekto $I$ sąsajos poaibis;
    TODO: \url{http://en.wikipedia.org/wiki/Subtype_polymorphism}
  \item[paveldėjimas, pavedimas] \en{inheritance, delegation} – galimybė
    perpanaudoti jau egzistuojantį kodą: tai gali būti pasiekiama
    objektų kūrimui naudojant klases, kurios gali paveldėti kai kurias
    savybes iš tėvinių klasių, arba naudojant žinučių persiuntimą;
    TODO: \url{http://en.wikipedia.org/wiki/Delegation_(programming)}
  \item[atvira rekursija] \en{open recursion} – specialaus kintamojo
    (dažniausiai jis vadinamas „this“, arba „self“) egzistavimas, kuriuo
    pasinaudojant galima kreiptis į kitus to paties objekto metodus.
    Svarbi savybė yra \en{late-bound}.
    TODO: \url{http://en.wikipedia.org/wiki/Name_binding}
    TODO: \url{http://en.wikipedia.org/wiki/Late_binding}
\end{description}

\section{Papildomos savybės}
