\appendix
\Chapter*{Priedai}
\def\thesection{\arabic{section} priedas.}
\section[\hspace{1.5em} Kodo pavyzdžiai]{Kodo pavyzdžiai}

\begin{scalainterpreterlisting}
  \inputscalai{AbstractMembers4}
  \ucaption{Parametrizuotų tipų panaudojimo atvejis}
  \label{lst:scala:abstract:members:4}
\end{scalainterpreterlisting}

\begin{scalacodelisting}
  \inputscala[lastline=22]{e1/Demo}
  \ucaption{Fragmentų hierarchijos modifikavimas}
  \label{lst:scala:mixin:1}
\end{scalacodelisting}

\begin{scalainterpreterlisting}
  \inputscalai{Mixin2}
  \ucaption{\ref{lst:scala:mixin:1} kodo pavyzdyje pateiktos hierarchijos
  panaudojimas}
  \label{lst:scala:mixin:2}
\end{scalainterpreterlisting}

\begin{scalacodelisting}
  \inputscala[lastline=32]{e10/Demo}
  \ucaption{Maišos kompozicijos panaudojimas}
  \label{lst:scala:mixin:3}
\end{scalacodelisting}

\begin{scalainterpreterlisting}
  \inputscalai{Mixin4}
  \ucaption{\ref{lst:scala:mixin:3} kodo pavyzdyje sukomponuotos
  klasės \scala{CachedFactorial} panaudojimas}
  \label{lst:scala:mixin:4}
\end{scalainterpreterlisting}

\begin{scalacodelisting}
  \inputscala[lastline=19]{e11/Demo}
  \ucaption{Savojo tipo anotacijos panaudojimas}
  \label{lst:scala:selftype:1}
\end{scalacodelisting}

\begin{scalainterpreterlisting}
  \inputscalai{Selftype2}
  \ucaption{\ref{lst:scala:selftype:1} kodo pavyzdyje apibrėžtų
  klasių panaudojimas}
  \label{lst:scala:selftype:2}
\end{scalainterpreterlisting}

\begin{scalainterpreterlisting}
  \inputscalai{Compiler1}
  \ucaption{Scala kompiliatoriaus objekto sukūrimas}
  \label{lst:scala:compiler:1}
\end{scalainterpreterlisting}
