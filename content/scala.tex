\Chapter{Scala}

Šiame skyriuje pristatyta programavimo kalba \plangname{Scala}:
nurodyta kuo ji skiriasi nuo \plangname{Java}, aprašytas
ir išanalizuotas jos komponentinis modelis.

\section{\plangname{Scala} ir \plangname{Java} skirtumai}

\plangname{Scala} kūrėjų \cite[1]{scala-overview} teigimu bent iš
dalies komponentinių technologijų evoliucijai trukdo programavimo
kalbų, kurios yra naudojamos komponentų kūrimui ir jų jungimui,
trūkumai. Dėl šios priežasties jie pabandė sukurti programavimo
kalbą, su kuria naudojant tas pačias priemones galima būtų aprašyti
tiek mažas tiek dideles dalis\footnote{\plangname{Scala} išsišifruoja,
kaip \emph{scalable language}.}. \plangname{Scala} yra multiparadigminė
programavimo kalba, derinanti objektinio ir funkcinio programavimo
savybes:
\begin{itemize}
  \item ji yra objektinė kalba ta prasme, kad kiekviena reikšmė
    \en{value} yra objektas ir kiekviena operacija yra metodo kvietimas
    \cite[3]{scala-overview};
  \item ji yra funkcinė kalba ta prasme, kad kiekviena funkcija yra
    reikšmė \en{value}.
\end{itemize}
Toliau yra pristatoma kuo \plangname{Scala} pakeitė statinį, klasinį
objektinį programavimą, kuriam „atstovauja“ \plangname{Java}.
\plangname{Scala}, kaip funkcinės kalbos savybės šiame darbe nėra
nagrinėjamos.

\plangname{Scala} autoriai \cite{scalable-component-abstractions}
išskyrė tris programavimo kalbos elementus įgalinančius kurti įvairaus
dydžio \en{scalable} komponentus:
\begin{enumerate}
  \item Abstraktūs tipai – nariai \en{abstract type members}.
  \item Savo tipo anotavimas \en{selftype annotations}.
  \item Modulinė maišos kompozicija \en{modular mixin composition}.
\end{enumerate}

\plangname{Scala}, be parametrizuojamųjų tipų \en{generics}, kuriuos
turi \plangname{Java}, turi dar vieną abstrakcijos mechanizmą:
\plangname{Scala} klasės gali turėti narius \en{member}, kurie
yra abstraktūs tipai. Abu mechanizmai yra pakeičiami vienas kitu:
\cite[10]{scala-overview} pateiktas pavyzdys, kaip parametrizuojamuosius
tipus galima būtų modeliuoti pasinaudojant abstrakčiais tipais, taip
pat nurodoma, kad yra įmanomas ir atvirkštinis variantas.
\plangname{Scala} kūrėjų \cite[11]{scala-overview} nurodytos
priežastys kalboje realizuoti abu mechanizmus yra skirtingi jų
panaudojimo atvejai: parametrizuojamus tipus yra siūloma naudoti kai
norime tiesiog galimybės klasės naudotojui nurodyti konkretų tipą
(tipinis pavyzdys būtų kolekcijos, kurių panaudojimas pademonstruotas
\ref{lst:scala:abstract:members:4} \plangname{Scala} sesijos
pavyzdyje\footnote{Kodo pavyzdžiai pateikti priede.}), o abstrakčius
tipus tada, kai norime kliento kode
pasinaudoti abstrakčiu tipu. Pastaroji galimybė yra dažniausiai
naudojama su kita \plangname{Scala} „naujove“\footnote{
Visi trys programavimo kalbos elementai, kurie \plangname{Scala}
kūrėjų teigimu yra reikalingi komponentų kūrimui, (abstraktūs
tipai – nariai, savo tipo anotavimas ir modulinė maišos kompozicija)
egzistavo dar iki \plangname{Scala} sukūrimo, bet \plangname{Scala}
yra pirmoji programavimo kalba, kurioje realizuoti jie visi
\cite[2]{scalable-component-abstractions}.} – moduline maišos
kompozicija.

% \label{lst:scala:abstract:members:4}

Modulinė maišos kompozicija, tai klasių kūrimo mechanizmas
pasinaudojant fragmentų \en{trait} komponavimu. Fragmentą galimą
būtų apibrėžti kaip sąsają, kurios metodai gali turėti
realizacijas. Šiuo požiūriu fragmentai primena abstrakčias klases,
tik skirtingai nuo jų, fragmentų konstruktoriai negali turėti
parametrų. Fragmento konstrukcija leidžia pasinaudoti
multipaveldėjimo suteikiamais kodo perpanaudojimo privalumais
(klasę galima sukomponuoti iš kelių fragmentų), bet tuo pačiu, dėl
to, kad kompiliavimo metu paveldėjimo grafas yra ištiesinamas, neturi
taip vadinamos rombo problemos.

Klasių hierarchijos pakeitimas į fragmentų hierarchiją, net ir
nesinaudojant komponentinio galimybėmis, turi privalumą, kad
gauta sistema yra lengviau modifikuojama. Komponuojant klasę iš
fragmentų yra įmanoma „įlįsti“ į fragmentų hierarchijos vidų
ir ten atlikti pakeitimus, ko negalima padaryti su klasių hierarchija,
nes pastarosios ištiesinimas turi tenkinti savybę: „klasės hierarchijos
ištiesinimas visada turi tiesioginės tėvinės klasės hierarchijos
ištiesinimą, kaip galūnę“\cite[57]{scala-reference}. Galimybę
atlikti pakeitimus viduje hierarchijos medžio iliustruojanti fragmentų
hierarchija pateikta \ref{lst:scala:mixin:1} kodo pavyzdyje,
o jos panaudojimas pateiktas \ref{lst:scala:mixin:2} \plangname{Scala}
sesijos pavyzdyje. Šiame pavyzdyje klasė \scala{C2} yra sukomponuojama
iš fragmentų \scala{M3} ir \scala{T4} pritaikant maišos kompoziciją.

% \label{lst:scala:mixin:1}
% \label{lst:scala:mixin:2}

Turbūt svarbiausia maišos kompozicijos savybė yra ta, kad konkretus
narys visada užkloja abstraktų
\cite[6]{scalable-component-abstractions}. Ši savybė leidžia
sujungti du fragmentus, kurių nesieja bendra hierarchija. Tai yra
iliustruota \ref{lst:scala:mixin:3} kodo pavyzdyje.
Sukomponuotosios klasės \scala{CachedFactorial} panaudojimas
pademonstruotas \ref{lst:scala:mixin:4} \plangname{Scala} sesijos
pavyzdyje. Fragmento \scala{CachedCalculation} nariai
\scala{KeyType}, \scala{ValueType} ir \scala{calculate} yra abstraktūs
ir jie yra užklojami atitinkamų narių apibrėžtų fragmente
\scala{Factorial}. Fragmente \scala{Factorial} taip pat yra
abstraktus metodas \scala{lookup}, kurį užkloja konkretus metodas
\scala{lookup}, apibrėžtas \scala{CachedCalculation}.

% \label{lst:scala:mixin:3}
% \label{lst:scala:mixin:4}

Taikant maišos kompoziciją kartais prireikia, kad
\scala{this}\footnote{\scala{this} yra specialus kintamasis rodantis į
objektą, kurio metodas dabar yra vykdomas.} tipas būtų ne tas fragmentas,
kurio metodas šiuo metu yra vykdomas, o kažkoks kitas, išvestinis
iš šio. Tai galima realizuoti vykdymo metu keičiant \scala{this}
tipą į norimą arba pasinaudojant \plangname{Scala} savojo tipo
anotacija. Pastarasis būdas turi privalumą, kad klaidos yra randamos
kompiliavimo metu. Savojo tipo anotacijos panaudojimas pateiktas
\ref{lst:scala:selftype:1} kodo pavyzdyje, o jame aprašytų klasių
veikimas iliustruotas \ref{lst:scala:selftype:2} \plangname{Scala}
sesijos pavyzdyje. \ref{lst:scala:selftype:1} kodo pavyzdyje savojo tipo
anotacija yra \scala{this: Node =>} – su ja nurodoma, kad
\scala{BaseNode} viduje \scala{this} tipas yra \scala{Node}. Todėl
kvietimas \scala{show(this)} tampa galimas.

% \label{lst:scala:selftype:1}
% \label{lst:scala:selftype:2}

Be savojo tipo anotacijos \ref{lst:scala:selftype:1} kodo pavyzdyje buvo
panaudotas dar vienas svarbus programavimo kalbos elementas – vidinė
klasė (ir fragmentas). \cite[12]{scalable-component-abstractions}
nurodo, kad jie šio elemento neįtraukė tarp būtinų komponentiniam
programavimui vien todėl, kad vidines klases palaiko pagrindinės
\en{mainstream} objektinės programavimo kalbos. Turbūt esminis
skirtumas tarp \plangname{Java} ir \plangname{Scala} vidinių klasių
yra tai, kad \plangname{Java} vidinės klasės tipas yra susietas su
išorine klase, o \plangname{Scala} su išoriniu objektu.
\ref{lst:scala:selftype:2} \plangname{Scala} sesijos pavyzdyje kintamojo
\scala{root} tipas yra \scala{tree.SimpleNode}, kai \plangname{Java} jis
būtų \scala{Tree.SimpleNode}. Ši savybė yra naudinga, pavyzdžiui, tuo, kad 
leidžia statiniais metodais užtikrinti, kad skirtingų abstrakčios
gamyklos \en{abstract factory} realizacijų objektai nebūtų maišomi
tarpusavyje \cite[36]{scala-design-patterns}. Prireikus šią savybę
galima apeiti pasinaudojant tipų projekcija \en{type projection},
tai yra išreikštinai nurodant, kad vidinės klasės objekto tipas
turi priklausyti nuo išorinės klasės. \ref{lst:scala:selftype:2}
sesijos pavyzdyje nurodyta, kad kintamojo \scala{node} tipas yra
bet kokio \scala{Tree} tipo objekto \scala{SimpleNode} tipo objektas.

\section{Scala komponentinis modelis}

\cite[37]{heineman2001component} komponentinį modelį apibrėžė kaip
standartų rinkinį. Žemiau paaiškinta, kaip \plangname{Scala}
komponentinis modelis realizuoja kiekvieną iš jų (kas aprašoma
standartuose trumpai paaiškinta \ref{section:component:model}
skyrelyje):
\begin{enumerate}
  \item \emph{Realizacija.}
    Komponentas gali būti realizuojamas, kaip \plangname{Scala} klasė
    arba kaip fragmentas.
  \item \emph{Sąsajos.}
    Komponentas servisus, kurių jam reikia, nurodo kaip abstrakčius
    narius, o kuriuos jis realizuoja – kaip konkrečius.
  \item \emph{Įvardinimas.}
    Komponentams globalūs vardai yra priskiriami taip pat, kaip ir
    \plangname{Java} klasėms – pagal hierarchines vardų sritis.
  \item \emph{Meta duomenys.} Kadangi \plangname{Scala} komponentai yra
    sujungiami statiškai kompiliavimo metu, tai meta duomenys nėra
    reikalingi.
  \item \emph{Tarpusavio sąveika.}
    Kadangi komponentų egzemplioriai \en{component instances} yra
    objektai, tai jie sąveikauja keisdamiesi žinutėmis.
  \item \emph{Pritaikymas.}
    Komponentą pritaikyti konkrečiam atvejui galima jam pateikiant
    kitokias jo abstrakčių narių realizacijas (pavyzdžiui, kitą
    konkretų tipą). Taip pat komponento egzemplioriaus sukūrimo
    metu jį galima pritaikyti pasinaudojant konstruktoriaus parametrais
    bei parametrizuojamais tipais.
  \item \emph{Kompozicija.}
    Komponentai yra surišami kompiliavimo metu. Abstraktūs nariai su
    jų realizacijomis yra sujungiami pagal sutampančią signatūrą.
    Kompozicijos rezultatas yra naujas komponentas, kuris gali būti
    toliau komponuojamas.
  \item \emph{Evoliucijos palaikymas.}
    Kadangi komponentai yra surišami kompiliavimo metu, tai norint
    pakeisti vieną komponentą kitu reikia perkompiliuoti sistemą.
    Taip pat nėra galimybės turėti dvi to paties komponento versijas
    vienoje sistemoje.
\end{enumerate}

\plangname{Scala} komponentinis modelis neapibrėžia kaip turėtų būti
supakuojami komponentai platinimui ir įdiegimui. Kadangi viena iš
esminių komponento savybių yra tai, kad jis yra fizinis
diegimo vienetas, tai galima būtų teigti, kad \plangname{Scala}
komponentinis modelis šios komponentinio modelio apibrėžimo dalies
netenkina.

Tam, kad \plangname{Scala} komponentinis modelis visiškai tenkintų
komponentinio modelio apibrėžimą, reikia apibrėžti komponentų
platinimo būdus. Problema tame, kad \plangname{Scala} komponentai
gali būti labai maži (kartais prasmingas komponentas gali būti tik
kelių dešimčių kodo eilučių ilgio) ir apibrėžti bendras taisykles
komponentų pakavimui ir įdiegimui gali būti tiesiog neprasminga,
nes tada komponento supakavimui greičiausiai reikės daugiau pastangų
nei jo sukūrimui. Nagrinėjant, kaip didesni komponentai galėtų
būti platinami, tai kaip vienas iš būdų platinti \plangname{Scala}
komponentus, būtų pasinaudojant versijų kontrolės sistemomis,
kadangi programų kodas dažniausiai yra laikomas jose. Pavyzdžiui,
\progname{Git}\footnote{\url{http://git-scm.com/}} ir
\progname{Mercurial}\footnote{\url{http://mercurial.selenic.com/}} palaiko
įdėtines saugyklas, taigi komponentų, iš kurių sudaryta sistema,
versijų kontrolės saugyklas galima būtų nurodyti, kaip įdėtines
kuriamos sistemos versijų kontrolės sistemai. Alternatyvus komponentų
platinimo būdas būtų pasinaudojant JAR failais\footnote{Nors Scala
komponentai yra sujungiami kompiliavimo metu, bet sujungti galima ir jau
sukompiliuotus komponentus.}. Taigi tokiu būdu papildžius
\plangname{Scala} komponentinį modelį gautume komponentinį modelį
atitinkantį \cite[37]{heineman2001component} pateiktą apibrėžimą.

\section{\plangname{Scala} komponentinio modelio savybės}

Šiame skyrelyje pateikiama \plangname{Scala} komponentinio modelio
analizė remiantis \ref{section:component:model:classification}
skyrelyje pristatyta \cite{classification-framework-for-scm}
klasifikacija.

\subsection{Gyvavimo ciklas}

\begin{enumerate}
  \item \emph{Modeliavimas.}
    Specialių metodų ar įrankių nėra. Iš dalies (be konstrukcijų
    perimtų iš funkcinių programavimo kalbų) galima modeliuoti
    pasinaudojant \progname{UML}\footnote{\url{http://uml.org/}}, ją
    papildžius tokiomis konstrukcijomis kaip fragmentas ir
    modulinė maišos kompozicija. Galima realizacija pateikiama
    \cite[145]{rachimow2009scala}.
  \item \emph{Realizacija.}
    Galima teigti, kad \plangname{Scala} modelis, kaip ir kiti
    \cite{classification-framework-for-scm} analizuoti komponentiniai
    modeliai yra labiausiai orientuotas į šią komponento gyvavimo
    ciklo dalį. \plangname{Scala} komponentinis modelis iš kitų
    bendro pobūdžio komponentinių modelių išsiskiria tuo, kad
    komponentų realizavimui jis pateikia ne tik taisykles, bet ir
    kalbos konstrukcijas. (Skirtingai, pavyzdžiui, nuo \progname{EJB}
    ar \progname{OSGi}, kurie tik apriboja programavimo su
    \plangname{Java} būdus.)
  \item \emph{Supakavimas.}
    Šios stadijos \plangname{Scala} komponentinis modelis nepalaiko.
  \item \emph{Įdiegimas.}
    Komponentai yra surišami kompiliavimo metu.
\end{enumerate}

\subsection{Konstravimas}

\begin{enumerate}
  \item \emph{Sąsajos.}
    \begin{enumerate}
      \item Sąsajos tipas yra paremtas operacijomis (metodų kvietimas).
      \item Teikiami ir prašomi servisai yra atskirti.
      \item Sąsajos yra aprašomos \plangname{Scala} programavimo
        kalba.
      \item Sąsajų suderinamumas yra tikrinamas tik sintaksiniu
        požiūriu.
    \end{enumerate}
  \item \emph{Saistymo mechanizmai.}
    \begin{enumerate}
      \item Komponentai yra sujungiami tiesiogiai. Tokie elementai,
        kaip jungtys, nėra išskiriami.
      \item Kadangi kelių komponentų kompozicijos rezultatas vėl yra
        komponentas, tai \plangname{Scala} komponentinis modelis
        palaiko vertikalų saistymą. Dėl to, kad \plangname{Scala}
        komponentinis modelis nenurodo kokius ekstra-funkcinius
        reikalavimus turi tenkinti komponentai, tai jis taip pat
        palaiko ir pilną vertikalią kompoziciją.
    \end{enumerate}
  \item \emph{Sąveika.}
    \begin{enumerate}
      \item Komponentų sąveikos stilius yra \emph{užklausa-atsakymas}.
      \item Komunikacija tarp komponentų yra sinchroninio tipo.
    \end{enumerate}
\end{enumerate}

\subsection{Ekstra-funkcinės savybės}

\begin{enumerate}
  \item \emph{Specifikavimas.}
    \plangname{Scala} komponentinis modelis ekstra-funkcinių savybių
    specifikavimo nepalaiko.
  \item \emph{Tvarkymas.}
    \plangname{Scala} komponentinis modelis nepateikia jokio
    palaikymo ekstra-funkcinių savybių tvarkymui, tuo turi
    pasirūpinti komponentų kūrėjai. (Metodas būtų \emph{vidinis
    per sąveiką}.)
  \item \emph{Kompozicija.}
    \plangname{Scala} komponentinis modelis ekstra-funkcinių savybių
    kompozicijos nepalaiko.
\end{enumerate}

\subsection{Išskirtinumai}

Iš kitų \cite{classification-framework-for-scm} nagrinėtų komponentinių
modelių \plangname{Scala} komponentinis išsiskiria tuo, kad
pilnai palaiko vertikalią kompoziciją. Iš to atsiranda privalumų
ir trūkumų. Kadangi komponentai, iš kurių yra surinktas ansamblis
yra paslepiami jo viduje, tai toks modelis labiau tinkamas realaus
pasaulio modeliavimui, nei modelis palaikantis tik horizontalų jungimą.
Pavyzdžiui, modeliuojant su \plangname{Scala} komponentas variklis po
kompozicijos būtų paslėptas komponente automobilis ir galima būtų
naudotis automobiliu nieko nežinant apie jo variklį. Tuo tarpu
modeliuojant su komponentiniu modeliu, kuris palaiko tik horizontalų
jungimą, komponentai variklis ir automobilis būtų atskirai, todėl
kiekvieną kartą kažką darant su automobiliu, reikėtų nepamiršti
ir jo variklio. Kitaip tariant \plangname{Scala} komponentais, kurie
yra ansambliai, yra žymiai paprasčiau naudotis, nei kitų modelių
ansambliais, kurie yra tiesiog komponentų rinkiniai. Pavyzdžiui,
nors \plangname{Scala} kompiliatorius yra ansamblis sukomponuotas
iš kelių skirtingų komponentų, bet norint jį inicijuoti nieko apie
tuos komponentus, iš kurių jis buvo sukomponuotas, žinoti nereikia:
pakanka susikurti kompiliatoriaus objektą pasinaudojant standartiniu
\scala{new} operatoriumi (\ref{lst:scala:compiler:1} \plangname{Scala}
sesijos pavyzdys). Be nurodyto privalumo, komponentų paslėpimas
ansamblyje turi ir trūkumą: kadangi komponentai yra paslepiami viduje,
tai norint juos pakeisti gali reikėti išardyti ansamblį.

% \label{lst:scala:compiler:1}

Be to, kad palaiko vertikalią kompoziciją \plangname{Scala}
komponentinis modelis išsiskiria dar ir tuo, jog yra realizuotas, kaip
programavimo kalba. Kitaip tariant, \plangname{Scala} yra bandymas
apjungti du komponentinių sistemų kūrimo lygius (komponentų kūrimo
ir jų tvarkymo) į vieną. Su tuo yra susiję keletas svarbių pastebėjimų:
\begin{enumerate}
  \item Apibrėžus komponentą, kaip įdiegimo vienetą, yra natūralu
    kalbėti apie komponentų programavimą ir sistemų surinkimą iš
    komponentų. Tuo tarpu, kadangi \plangname{Scala} komponentai yra
    sujungiami programuojant, tai šiuo atveju galima kalbėti jau ir
    apie komponentinį programavimą.
  \item Su \plangname{Scala} galima kurti net labai mažus komponentus:
    pavyzdžiui, \progname{EJB} mažiausias komponentas yra
    \plangname{Java} klasė, tuo tarpu \plangname{Scala} mažiausias
    komponentas yra fragmentas, o tai yra mažesnis darinys už klasę.
  \item Kuriant sistemą su \plangname{Scala}, yra vartojamos tos
    pačios sąvokos (objektas, fragmentas, klasė, modulinė maišos
    kompozicija ir t.t.), nepriklausomai nuo to kokio dydžio
    sistema yra kuriama.
  \item \cite[36]{heineman2001component} teigimu komponentų vardai
    negali būti naudojami, kaip tipų vardai. Tam gali būti naudojami
    tik sąsajų vardai. \plangname{Scala} šito apribojimo neturi.
  \item Su \plangname{Scala} yra lengviau kurti sudėtingas sistemas,
    turinčias rekursyvias priklausomybes tarp komponentų nei su
    objektinėms sistemoms skirtais komponentiniais modeliais, nes
    priklausomybės yra automatiškai ištiesinamos ir patikrinamos
    kompiliavimo metu, o komponentus realizuojant objektinėmis
    programavimo kalbomis tektų arba taikyti metaprogramavimą,
    arba rašyti daug palaikančio kodo
    \cite[9]{scalable-component-abstractions}.
  \item Kadangi norint išmokti kurti komponentines sistemas su
    \plangname{Scala} tereikia įsisavinti vieną technologiją
    vietoj dviejų (programavimo kalbos ir komponentinio modelio),
    tai su ja yra lengviau išmokti kurti sistemas komponentiškai.
  \item \plangname{Scala} yra suderinama su kitomis komponentinėmis
    technologijomis (su \plangname{Scala} galima realizuoti jų
    komponentus), kurios palaiko \plangname{Java} (pavyzdžiui,
    \progname{OSGi}, \progname{EJB}). Tai leidžia apeiti kai kuriuos
    \plangname{Scala} komponentinio modelio trūkumus: pavyzdžiui,
    jis nesuteikia priemonių leidžiančių realizuoti sistemas, kurių
    komponentai „gyventų“ skirtingose fizinėse mašinose.
\end{enumerate}

Apibendrinant galima būtų teigti, kad \plangname{Scala} komponentinio
modelio esminis privalumas, lyginant su kitais, yra jo paprastumas.
