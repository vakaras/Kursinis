\chapter{Scala}

\section{Tikslas}

Išsiaiškinti, kaip galima būtų padidinti verslo palaikymo sistemų, kurtų
naudojant klasikines objektines programavimo kalbas, modifikuojamumą.

Sistemos modifikuojamumas suprantamas, kaip galimybė vystyti sistemą
evoliuciniu būdu: sistemą papildyti naujomis funkcijomis neliečiant
senųjų. Priešprieša evoliuciniam sistemos vystymo būdui būtų
revoliucinis: norint pridėti naują funkciją, reikia keisti jau
esamas.

Šiame darbe laikoma, kad klasikinė objektinė programavimo kalba
yra statinė klasinė objektinė programavimo kalba. Kaip pagrindinis
pavyzdys yra nagrinėjama Java.

\section{Uždaviniai}

\begin{enumerate}
  \item Išsiaiškinti, kuo klasikinį objektinį papildė Scala
    kūrėjai ir kokias papildomas galimybes suteikia pridėtosios
    savybės.
  \item Išskirti kokias savybes pridedant klasikiniam objektiniam
    galima padidinti jo modifikuojamumą.
  \item Išsiaiškinti, kaip reikėtų programuoti tam, kad pasinaudoti
    papildomų savybių suteikiamomis galimybėmis.
\end{enumerate}

\section{Scala išskirtinės savybės}

Straipsnyje \cite{scalable-component-abstractions} Martin Odersky
ir Matthias Zenger išskyrė tris savybes įgalinančias kurti
įvairaus dydžio \en{scalable} komponentus:
\begin{itemize}
  \item abstraktūs tipai \en{abstract type members};
  \item atviros rekursijos anotacija \en{selftype annotations, open
    recursion};
  \item modulinė maišos kompozicija \en{modular mixin composition}.
\end{itemize}

Ką jie turi omeny sakydami įvairaus dydžio komponentus matosi
iš tokio pavyzdžio.
TODO: Pavyzdys, kaip galima perdengti nested klasę.

\subsection{Kodo dubliavimo vengimas}

Programuojant objektinėmis kalbomis, kurios neturi multipaveldėjimo
yra susiduriama su problema, jog kai kuriais atvejais yra sudėtinga
išvengti kodo dubliavimo. Panagrinėsime du pavyzdžius.

\subsubsection{Trečiųjų šalių bibliotekos kodo modifikavimas}

TODO: Pavyzdys, kaip pasinaudojant Scala savybės \en{trait} konstrukcija
galima išvengti kodo dubliavimo taisant trečiųjų šalių biblioteką.

\subsubsection{Kraunamų savybių dizaino šablonas}

\en{Stackable trait design pattern}\cite[267p.]{programming-in-scala}.

TODO: (e1)

\section{Scala komponentinis modelis}

Straipsnyje \cite{scalable-component-abstractions} Martin Odersky
ir Matthias Zenger išskyrė tris programavimo kalbos elementus
įgalinančius kurti įvairaus dydžio \en{scalable} komponentus:
\begin{itemize}
  \item abstraktūs tipai \en{abstract type members};
  \item atviros rekursijos anotacija \en{selftype annotations, open
    recursion};
  \item modulinė maišos kompozicija \en{modular mixin composition}.
\end{itemize}
Visi šie elementai yra realizuoti
Scala\footnote{\url{http://www.scala-lang.org}} programavimo kalboje.

Scala, vietoj klasinėms objektinėms programavimo kalboms įprastos
sąsajos \en{interface} konstrukcijos, turi savybės \en{trait}
konstrukciją. Pagrindinis savybės ir sąsajos skirtumas yra tai, kad
savybė gali turėti konkrečias metodų realizacijas. Taigi Scala
savybė labiau primena abstrakčią klasę, tik skirtingai nuo klasės,
Scala savybių konstruktoriai neturi parametrų. Savybės konstrukcija
leidžia pasinaudoti multi paveldėjimo suteikiamais kodo perpanaudojimo
privalumais (į klasę galima sukomponuoti kelias savybes), bet tuo
pačiu, dėl to, kad kompiliavimo metu paveldėjimo grafas yra
ištiesinamas, neturi taip vadinamos rombo problemos.

Pasinaudojant Scala savybės konstrukcija su Scala galima programuoti
komponentiškai:
\begin{itemize}
  \item komponentas realizuojamas pasinaudojant Scala klasės arba
    savybės konstrukcija;
  \item komponentas savo poreikius nurodo per abstrakčius metodus;
  \item komponentas savo teikiamus servisus nurodo per savo konkrečius
    metodus;
  \item komponentai yra surišami kompiliavimo metu;
  \item komponentų kompozicijos rezultatas yra naujas komponentas
    (Scala klasė).
\end{itemize}
Kadangi komponentų kompozicijos rezultatas yra nauja klasė, tai yra
komponentas, tai ją galima toliau jungti su kitais komponentais.
Kitaip tariant Scala komponentinis modelis palaiko vertikalų
jungimą
\en{vertical binding}\cite[598p.]{classification-framework-for-scm}.
Tiesa, jis nėra visiškai pilnas, nes negalima sukomponuoti dviejų
kompozicijos rezultatų (klasių) į naują komponentą.

Klasių hierarchijos pakeitimas į savybių hierarchiją turi privalumą,
kad kompozicijos metu galima „įlįsti“ į savybių hierarchijos vidų ir ten
atlikti modifikacijas, ko negalima padaryti su klasių hierarchija,
nes pastarosios ištiesinimas turi tenkinti savybę: „klasės hierarchijos
ištiesinimas visada turi tiesioginės tėvinės klasės hierarchijos
ištiesinimą, kaip galūnę“\cite[57p.]{scala-reference}. Toks sprendimas
priimtas todėl, kad galima būtų naudoti jau sukompiliuotas klases
naujame kode.

\begin{exmp}
  Pakeisdami klasių hierarchiją į savybių hierarchiją padidiname
  sistemos modifikuojamumą, nes iš savybių rinkinio kurdami naują
  klasę skirtingai nei klasių paveldėjimo atveju, galime įterpti
  naują savybę į norimą vietą. Tai iliustruoja toks kodo fragmentas:

  \inputscala{e1/Demo}

  Jį įvykdžius į standartinę išvestį bus atspausdinta:

  \begin{textcode}
    List(C1, T4, T3, T2, T1)
    List(C2, T4, T3, M3, T2, T1)
  \end{textcode}

\end{exmp}

Be trijų aukščiau išvardintų savybių Scala kūrėjai, kaip vieną iš
itin svarbių programavimo kalbos savybių reikalingų komponentų kūrimui
išskiria įdėtines klases. Jie jos neįtraukė tarp tų trijų
savybių vien todėl, kad dauguma pagrindinių programavimo kalbų
šią savybę turi\cite[12p.]{scalable-component-abstractions}.
Galimybė naudoti įdėtines klases yra svarbi tuo, kad jos gali pasiekti
tėvinės klasės atributus be jokio papildomo surišimo tuo pačiu
išvengiant klasės vardų srities užteršimo problemos. (Jei bandytume
apsieiti be įdėtinių klasių, tai greičiausiai būtume priversti
rašyti ilgesnius metodų pavadinimus, pavyzdžiui, „loggerAppend“,
„SyntaxTreeAppend“.) Scala patobulina įdėtines klases leisdama jų
paveldėjimą. (TODO: Patikrinti, ar Java to neturi.)

TODO: Pavyzdys su įdėtinėmis klasėmis:
class A {
  class Logger {}
  def getLogger(): Logger = new Logger()
}
class B {
  class ExtendedLogger extends Logger {}
  override def getLogger(): ExtendedLogger = new ExtendedLogger()
}
TODO: Patikrinti ar tai veikia Java?

Trait yra tai, ką galime sukomponuoti į klasę. Klasės komponavimo
metu mes nusprendžiame iš kokių savybių mes norime ją sudaryti.
Po sukomponavimo mes gauname standartinę Java klasę, kurią
toliau galime plėsti per numatytus plėtimo taškus. Trait naudojimas
leidžia pasinaudoti multipaveldėjimo teikiama nauda, bet apsaugo
nuo situacijos, kai bandoma sujungti dvi „nesuderinamas“ klases
(TODO: C++ pavyzdys su dviem nesuderinamomis klasėmis), nes
trait neturi konstruktoriaus ir jei jo nėra bandoma emuliuoti,
tai traitų būsena yra suderinama klasės sukomponavimo metu. Šis
modelis yra lengvai plečiamas, nes jei klasė traitams kitus traitus
perduoda pasinaudodama def konstrukcija, tai tada ją galime perdengti
išvestinėje klasėje.

Sukomponuotos klasės jau nebegalima keisti. (Lįsti į jos vidų. Ją galima
tik „anotuoti“.) Klases visur galima keisti į savybes. Klasės reikalingos
tik tada, kai mums reikia sukonstruoti naują objektą. Šis dizainas
lemia, kad egzistuoja tik viena vieta, kur yra sukonstruojamas objektas.

Taip pat šis komponentinis modelis palaiko vertikalų jungimą
\en{vertical binding}\cite[598p.]{classification-framework-for-scm}.
TODO: Įrodyti formaliai. Neformalus įrodymas: Scala palaiko vertikalų
jungimą, nes savybes sukomponavus į klasę, mes toliau galime tą
klasę komponuoti su naujomis savybėmis ir gauti naujas klases.
Visgi Scala vertikalus jungimas nėra visiškai pilnas, nes negalima
sukomponuoti dviejų ansamblių (klasių) į naują ansamblį.
