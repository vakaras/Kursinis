\chapter{Scala}

\section{Tikslas}

Išsiaiškinti, kaip galima būtų padidinti verslo palaikymo sistemų, kurtų
naudojant klasikines objektines programavimo kalbas, modifikuojamumą.

Sistemos modifikuojamumas suprantamas, kaip galimybė vystyti sistemą
evoliuciniu būdu: sistemą papildyti naujomis funkcijomis neliečiant
senųjų. Priešprieša evoliuciniam sistemos vystymo būdui būtų
revoliucinis: norint pridėti naują funkciją, reikia keisti jau
esamas.

Šiame darbe laikoma, kad klasikinė objektinė programavimo kalba
yra statinė klasinė objektinė programavimo kalba. Kaip pagrindinis
pavyzdys yra nagrinėjama Java.

\section{Uždaviniai}

\begin{enumerate}
  \item Išsiaiškinti, kuo klasikinį objektinį papildė Scala
    kūrėjai ir kokias papildomas galimybes suteikia pridėtosios
    savybės.
  \item Išskirti kokias savybes pridedant klasikiniam objektiniam
    galima padidinti jo modifikuojamumą.
  \item Išsiaiškinti, kaip reikėtų programuoti tam, kad pasinaudoti
    papildomų savybių suteikiamomis galimybėmis.
\end{enumerate}

\section{Scala išskirtinės savybės}

Straipsnyje \cite{scalable-component-abstractions} Martin Odersky
ir Matthias Zenger išskyrė tris savybes įgalinančias kurti
įvairaus dydžio \en{scalable} komponentus:
\begin{itemize}
  \item abstraktūs tipai \en{abstract type members};
  \item atviros rekursijos anotacija \en{selftype annotations, open
    recursion};
  \item modulinė maišos kompozicija \en{modular mixin composition}.
\end{itemize}

Ką jie turi omeny sakydami įvairaus dydžio komponentus matosi
iš tokio pavyzdžio.
TODO: Pavyzdys, kaip galima perdengti nested klasę.

\subsection{Kodo dubliavimo vengimas}

Programuojant objektinėmis kalbomis, kurios neturi multipaveldėjimo
yra susiduriama su problema, jog kai kuriais atvejais yra sudėtinga
išvengti kodo dubliavimo. Panagrinėsime du pavyzdžius.

\subsubsection{Trečiųjų šalių bibliotekos kodo modifikavimas}

TODO: Pavyzdys, kaip pasinaudojant Scala savybės \en{trait} konstrukcija
galima išvengti kodo dubliavimo taisant trečiųjų šalių biblioteką.

\subsubsection{Kraunamų savybių dizaino šablonas}

\en{Stackable trait design pattern}\cite[267p.]{programming-in-scala}.

TODO: (e1)
