\chapter{Žodynėlis}

\begin{glossary}

  \begin{entry}%
    {dynamic-dispatch}%
    {dinaminis susiejimas}%
    [dynamic dispatch]%
    [\url{http://en.wikipedia.org/wiki/Dynamic_dispatch}]

    Nusprendimo, kokį kodą vykdyti gavus žinutę, procesas,
    atliekamas programos vykdymo metu.

  \end{entry}

  \begin{entry}%
    {statically-typed-programming-language}%
    {statiškai tipizuota programavimo kalba}%
    [statically typed programming language]%
    [\url{http://en.wikipedia.org/wiki/Static_typing\#Static_typing}]

    Programavimo kalba, kurioje tipų patikrinimas yra atliekamas
    kompiliavimo metu.
    
  \end{entry}

  \begin{entry}%
    {dynamically-typed-programming-language}%
    {dinamiškai tipizuota programavimo kalba}%
    [dynamically typed programming language]%
    [\url{http://en.wikipedia.org/wiki/Dynamic_typing\#Dynamic_typing}]

    Programavimo kalba, kurioje tipų patikrinimas yra atliekamas vykdymo
    metu. Dinamiškai tipizuotose kalbose tipus turi reikšmės, o ne
    kintamieji.
    
  \end{entry}

  \begin{entry}%
    {polymorphism}%
    {polimorfismas}%
    [polymorphism]%
    [%
    \url{http://en.wikipedia.org/wiki/Polymorphism_(computer_science)}, %
    \url{http://en.wikipedia.org/wiki/Polymorphism_in_object-oriented_programming}%
    ]%

    Programavimo kalbos savybė, kuri leidžia su skirtingų tipų duomenimis
    dirbti naudojantis ta pačia sąsaja. Išskiriami trys polimorfizmo tipai:
    \begin{itemize}
      \item \emph{Ad-hoc} polimorfizmas \en{Ad-hoc polymorphism} –
        iš esmės, tai yra funkcijų perdengimas;
      \item parametrinis polimorfizmas \en{parametric polymorphism} –
        programuojama taip, kad kodas nepriklausytų nuo duomenų tipo
        (pavyzdys būtų C++ šablonai ir Java generics);
      \item potipio polimorfizmas \en{subtype polymorphism} – 
        jei $T$ yra $S$ potipis, tai $T$ galima naudoti vietoje $S$.
    \end{itemize}
    
  \end{entry}
\end{glossary}

\begin{description}

  \item[metodo užklotis \en{method overriding}]
    Objektinės programavimo kalbos savybė, kuri leidžia paveldinčiai klasei
    realizuoti metodą, kurį jau yra realizavusi kažkuri iš jos tėvinių
    klasių. Paveldinčios klasės metodo realizacija užkloja (paslepia)
    tėvinės klasės metodą.

  \item[funkcijos perdengimas \en{function overloading}]
    Programavimo kalbos savybė, kuri leidžia aprašyti kelias funkcijas
    turinčias tą patį vardą, kurios yra atskiriamos pagal jų argumentų
    tipus.

  \item[įdėtinė klasė \en{nested class, static inner class}]
    Java kalboje, tai klasė apibrėžta kitos klasės viduje, bet
    kurios objekto sukūrimui nėra būtinas gaubiančiosios
    klasės objektas. Ji apibrėžiama su \verb|static| konstrukcija.
    Scala neturi įdėtinių klasių.

  \item[vidinė klasė \en{inner class}]
    Java kalboje, tai klasė apibrėžta kitos klasės viduje. Jos
    objekto sukūrimui yra būtinas gaubiančiosios klasės objektas.
    Scala kalboje visos klasės apibrėžtos kitų klasių viduje
    yra vidinės klasės.

\end{description}
