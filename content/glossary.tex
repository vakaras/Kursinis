\chapter{Žodynėlis}

\begin{description}

  \item[dinaminis susiejimas \en{dynamic dispatch}]
    Nusprendimo, kokį kodą vykdyti gavus žinutę, procesas,
    atliekamas programos vykdymo metu.

    \url{http://en.wikipedia.org/wiki/Dynamic_dispatch}

  \item[statiškai tipizuota programavimo kalba \en{statically typed
    programming language}] 
    Programavimo kalba, kurioje tipų patikrinimas yra atliekamas kompiliavimo
    metu.

    \url{http://en.wikipedia.org/wiki/Static_typing#Static_typing}

  \item[dinamiškai tipizuota programavimo kalba \en{dynamically typed
    programming language}]
    Programavimo kalba, kurioje tipų patikrinimas yra atliekamas vykdymo
    metu. Dinamiškai tipizuotose kalbose tipus turi reikšmės, o ne
    kintamieji.

    \url{http://en.wikipedia.org/wiki/Dynamic_typing#Dynamic_typing}

  \item[polimorfizmas \en{polymorphism}]
    Programavimo kalbos savybė, kuri leidžia su skirtingų tipų duomenimis
    dirbti naudojantis ta pačia sąsaja. Išskiriami trys polimorfizmo tipai:
    \begin{itemize}
      \item \emph{Ad-hoc} polimorfizmas \en{Ad-hoc polymorphism} –
        iš esmės, tai yra funkcijų perdengimas;
      \item parametrinis polimorfizmas \en{parametric polymorphism} –
        programuojama taip, kad kodas nepriklausytų nuo duomenų tipo
        (pavyzdys būtų C++ šablonai ir Java generics);
      \item potipio polimorfizmas \en{subtype polymorphism} – 
        jei $T$ yra $S$ potipis, tai $T$ galima naudoti vietoje $S$.
    \end{itemize}

    \url{http://en.wikipedia.org/wiki/Polymorphism_(computer_science)}

    \url{http://en.wikipedia.org/wiki/Polymorphism_in_object-oriented_programming}

  \item[metodo užklotis \en{method overriding}]
    Objektinės programavimo kalbos savybė, kuri leidžia paveldinčiai klasei
    realizuoti metodą, kurį jau yra realizavusi kažkuri iš jos tėvinių
    klasių. Paveldinčios klasės metodo realizacija užkloja (paslepia)
    tėvinės klasės metodą.

  \item[funkcijos perdengimas \en{function overloading}]
    Programavimo kalbos savybė, kuri leidžia aprašyti kelias funkcijas
    turinčias tą patį vardą, kurios yra atskiriamos pagal jų argumentų
    tipus.

\end{description}
