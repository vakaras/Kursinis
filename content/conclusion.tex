\Chapter*{Rezultatai ir išvados}

% Rezultatų ir išvadų dalyje turi būti aiškiai išdėstomi
% pagrindiniai darbo rezultatai (kažkas išanalizuota, kažkas sukurta,
% kažkas įdiegta) ir pateikiamos išvados (daromi nagrinėtų problemų
% sprendimo metodų palyginimai, teikiamos rekomendacijos, akcentuojamos
% naujovės).

\TODO{Pademonstruoti, kad \plangname{Scala} turi modifikuojamumą,
ir praplečiamumą. Tada bus akivaizdesnis išvadų pilnumas:
\plangname{Scala} turi modifikuojamumą ir praplečiamumą, bet neturi
perpanaudojamumo, o \progname{Debian} turi perpanaudojamumą, bet neturi
modifikuojamumo ir praplečiamumo.}

Šiame darbe buvo pristatyta, kas yra komponentinis modelis, išskirtos
komponento savybės bei parodyta, kad \plangname{Scala} programavimo
kalba ir \progname{Debian} \gls{package-management-system}{paketų
tvarkymo sistema} yra komponentinės technologijos.

Nors objektinis programavimas ir komponentinis sistemų kūrimo būdas
yra du skirtingi programų sistemų kūrimo lygiai, bet
\plangname{Scala} kūrėjams pavyko šiuos lygius sujungti į vieną,
sukuriant kalbą, kuria yra įmanoma programuoti komponentiškai.
Pagrindinis \plangname{Scala} privalumas lyginant su kitomis
komponentinėmis technologijomis – jos naudojimo paprastumas. Taip pat,
\plangname{Scala} turi ir esminį trūkumą: tai, kad nėra apibrėžtas
\plangname{Scala} komponentų perdavimo formatas, apsunkina jos kodo
perpanaudojamumą.

Iš \progname{Debian} paketų tvarkymo sistemos ir egzistuojančių
komponentinių modelių analizės galima būtų teigti, kad
perpanaudojamumą pagerina komponentų saugyklų egzistavimas bei galimybė
diegiant komponentą automatiškai gauti visus jo funkcionavimui
reikiamus komponentus. Taip pat remiantis \progname{Debian} paketų
tvarkymo sistemos analize galima būtų kelti hipotezę, kad
sistemos modifikuojamumą padidina komponentų priklausymas tik nuo
abstrakcijų, o ne nuo konkrečių kitų komponentų.

Apibendrinant, galima būtų kelti hipotezę, kad norint, jog
komponentinis modelis pasižymėtų perpanaudojamumu, modifikuojamumu ir
praplečiamumu, reikia, kad jo komponentai be išimčių turėtų visas
\ref{section:component:descriptions} skyrelyje nurodytas komponento
savybes ir komponentinis modelis papildomai apibrėžtų komponentų
platinimo bei jų automatinio įdiegimo kartu su priklausomybėmis
mechanizmus.
