\chapter{Objektinis programavimas}

\section{Objektinio programavimo apibrėžimai}

TODO:
\begin{itemize}
  \item Objektinio programavimo apibrėžimas:
    \begin{itemize}
      \item dinaminis susiejimas;
      \item inkapsuliacija;
      \item potipiai;
      \item paveldėjimas, pavedimas;
      \item atvira rekursija.
    \end{itemize}
  \item Objektinio programavimo rūšys ir kur patartina ką naudoti:
    \begin{itemize}
      \item prototipinis objektinis;
      \item dinaminis klasinis objektinis;
      \item statinis klasinis objektinis.
    \end{itemize}
\end{itemize}

\section{Statinis klasinis objektiškai orientuotas programavimas}

TODO: Apibrėžimas ir nurodymas, kad toliau darbe apsiribojama tik juo.
Kaip pagrindinis pavyzdys nagrinėjama Java programavimo kalba.
