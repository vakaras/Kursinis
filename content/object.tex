\chapter{Objektinis programavimas}

Norint atskirti dvi paradigmas, reikia tiksliai apibrėžti kas priklauso
kiekvienai iš jų. Šiame skyriuje bus pabandyta išskirti kas yra
objektinis programavimas apskritai, kokios yra objektinio
rūšys\NTODO{Nagrinėjant Scala yra svarbu nurodyti ant kokio objektinio
ji „stovi“.} bei kam tos rūšys yra reikalingos (kokios yra jų
pritaikymo sritys)\NTODO{Nagrinėjant Scala panagrinėti ar ji keičia
pritaikymo sritį.}.

\section{Objektinio programavimo apibrėžimas}

Objektinis programavimas bendrąja prasme yra programavimas
pasinaudojant objekto abstrakcija. Objektą galima apibrėžti, kaip
\quote{esybę, kuri apjungia duomenų ir procedūrų savybes tam,
kad galėtų atlikti veiksmus ir turėti
būseną}\cite[41]{OOP-themes-and-variations}. Objektinės sistemos veikimas
yra suvokiamas, kaip objektų tarpusavio sąveika: objektai vienas kitam
siuntinėja žinutes, o kiekvienas gavęs žinutę objektas į ją sureaguoja
įvykdydamas atitinkamos procedūros (metodo)
kodą\cite[41]{OOP-themes-and-variations}%
\cite[277]{concepts-in-programming-languages}%
\cite[168]{Wegner:1987:DOL:38807.38823}.
Be būsenos ir veiksenos, prie objekto išskirtinių savybių 
\cite[37]{cs-beyond-object-oriented-programming} bei
populiariausių objektinių programavimo kalbų kūrėjai nurodo
dar ir tai, kad kiekvienas objektas turi unikalią tapatybę.
Taigi apibendrinant, objektai yra esybės, kurios:
\begin{enumerate}
  \item turi unikalią tapatybę;
  \item turi būsenas;
  \item sąveikauja besikeisdamos žinutėmis;
  \item reaguoja į žinutes įvykdydamos tam tikrą procedūrą.
\end{enumerate}

Siekiant išskirti
\gls{object-oriented-programming-language}{objektines programavimo
kalbas} iš programavimo kalbų, kuriomis įmanoma programuoti
objektiškai (pavyzdžiui, tarę, kad programavimo kalbos \plangname{Modula-2}
modulis yra objektas, su ja galėtume programuoti objektiškai), yra
reikalaujama, kad programavimo kalba, kurią vadiname objektine
palaikytų objektinį programavimą kalbinėmis priemonėmis. 1987
metais ACM OOPSLA konferencijos metu paskelbtame „Orlando
susitarime“ \en{„The Treaty of Orlando“}
\cite{Lieberman:1987:TO:62139.62144} nurodoma, kad esminis
objektinių programavimo kalbų bruožas yra 
\gls{object-description-sharing-mechanism}{dalinimosi \en{sharing}
mechanizmas} – galimybė perpanaudoti esamų objektų apibrėžimų dalis
apibrėžiant naujus objektus.
\gls{object-oriented-programming-language}{Objektinės programavimo kalbos}
tai gali įgyvendinti panaudodamos \gls{delegation}{delegavimo} arba
\gls{class-inheritance}{paveldėjimo} mechanizmus. Taigi, Modula-2 nėra
objektinė programavimo kalba, nes jos modulis negali automatiškai
„perimti“ kito modulio veiksenos.

Be šios „Orlando susitarime“ nurodytos savybės, dauguma
objektinių programavimo kalbų turi ir daugiau bendrų bruožų,
kuriuos išskyrė \cite[225-227]{types-and-programming-languages}:
\begin{description}
  \item[dinaminis susiejimas] \en{dynamic dispatch, dynamic binding} –
    objekto reakcija į gautą žinutę yra nustatoma vykdymo metu;
  \item[uždarumas, inkapsuliacija] \en{encapsulation} – vidinė objekto
    struktūra yra slepiama;
  \item[potipiai] \en{subtyping} – kai norime pasinaudoti objektu, 
    tai mums rūpi tik jo sąsaja ir mes galime naudoti objektą $I$ vietoj
    objekto $J$, jei objekto $J$ sąsaja yra objekto $I$ sąsajos poaibis;
  \item[paveldėjimas, pavedimas] \en{inheritance, delegation} – galimybė
    perpanaudoti jau egzistuojantį kodą: tai gali būti pasiekiama
    objektų kūrimui naudojant klases, kurios gali paveldėti kai kurias
    savybes iš tėvinių klasių, arba naudojant žinučių persiuntimą;
  \item[atvira rekursija] \en{open recursion} – specialaus kintamojo
    (dažniausiai jis vadinamas „this“, arba „self“) egzistavimas, kuriuo
    pasinaudojant galima kreiptis į kitus to paties objekto metodus.
\end{description}
Nors paminėtos savybės ir yra bendros populiariausioms objektinėms
programavimo kalboms, bet jos jas įgyvendina įvairiais būdais,
dėl ko jos įgauna savitumų, kurie daro įtaką šių programavimo
kalbų pritaikymo sritims.

\section{Objektinių kalbų rūšys}

Įvertinant tai, kad objektinių programavimo kalbų yra ne viena,
galima kelti hipotezę, kad neegzistuoja objektinės programavimo
kalbos savybių rinkinys, kuris tiktų visoms situacijoms. Todėl
atsiranda poreikis nustatyti:
\begin{enumerate}
  \item kokios kalbos savybės kokiems atvejams yra labiau tinkamos;
  \item kurios iš tų savybių yra suderinamos, o kurios ne.
\end{enumerate}
Vienas iš itin rūpimų kriterijų yra galimybė sistemas kurti evoliuciniu
būdu, tai yra mums rūpi kokios kalbos savybės leidžia evoliuciniu
būdu kurti sistemas skirtingomis situacijomis:
\begin{itemize}
  \item kai aplinka yra nežinoma arba labai greitai kintanti;
  \item kai sistema kinta palyginus lėtai.
\end{itemize}
Kiti du svarbūs kriterijai: sistemos tobulinimo kaina bei sistemos
stabilumas.

„Orlando susitarime“ išskiriami iš principo du skirtingi objektinių
kalbų pritaikymo atvejai:
\begin{enumerate}
  \item dalykinė sritis yra labai greitai kintanti – prioritetas yra
    galimybė greitai atlikti pakeitimus;
  \item dalykinė sritis kinta lėtai – prioritetas yra garantijos,
    kad sistema veikia taip, kaip numatyta.
\end{enumerate}

\section{Objektinio programavimo rūšių taikymai}

Prototipinė objektinė programavimo kalba (\plangname{JavaScript}) yra
tinkamesnė žmonių kalbos modeliavimui nei klasinė programavimo
kalba, nes žmonių kalboje paprastai yra daugiau išimčių negu
taisyklių.

\section{Objektinio programavimo apibrėžimai}

TODO:
\begin{itemize}
  \item Objektinio programavimo apibrėžimas:
    \begin{itemize}
      \item dinaminis susiejimas;
      \item inkapsuliacija;
      \item potipiai;
      \item paveldėjimas, pavedimas;
      \item atvira rekursija.
    \end{itemize}
  \item Objektinio programavimo rūšys ir kur patartina ką naudoti:
    \begin{itemize}
      \item prototipinis objektinis;
      \item dinaminis klasinis objektinis;
      \item statinis klasinis objektinis.
    \end{itemize}
\end{itemize}

\section{Statinis klasinis objektiškai orientuotas programavimas}

TODO: Apibrėžimas ir nurodymas, kad toliau darbe apsiribojama tik juo.
Kaip pagrindinis pavyzdys nagrinėjama Java programavimo kalba.
