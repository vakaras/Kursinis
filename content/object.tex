\chapter{Objektinis programavimas}

Norint atskirti dvi paradigmas, reikia tiksliai apibrėžti kas priklauso
kiekvienai iš jų. Šiame skyriuje pabandyta išskirti kas yra
objektinis programavimas apskritai, kokios yra objektinio
rūšys\NTODO{Nagrinėjant Scala yra svarbu nurodyti ant kokio objektinio
ji „stovi“.} bei kam tos rūšys yra reikalingos (kokios yra jų
pritaikymo sritys)\NTODO{Nagrinėjant Scala panagrinėti ar ji keičia
pritaikymo sritį.}.

\section{Objektinio programavimo apibrėžimas}

\cite[225]{types-and-programming-languages} teigimu yra beprasmiška
bandyti tiksliai apibrėžti ką reiškia „objektiškai orientuotas“,
bet objektinį programavimą bendrąja prasme galime suvokti, kaip
programavimą pasinaudojant objekto abstrakcija. Objektą galima
apibrėžti, kaip \quote{esybę, kuri apjungia duomenų ir procedūrų
savybes tam, kad galėtų atlikti veiksmus ir turėti
būseną}\cite[41]{OOP-themes-and-variations}. Objektinės sistemos veikimas
yra suvokiamas, kaip objektų tarpusavio sąveika: objektai vienas kitam
siuntinėja žinutes, o kiekvienas gavęs žinutę objektas į ją sureaguoja
įvykdydamas atitinkamos procedūros (metodo)
kodą\cite[41]{OOP-themes-and-variations}%
\cite[277]{concepts-in-programming-languages}%
\cite[168]{Wegner:1987:DOL:38807.38823}.
Be būsenos ir veiksenos, prie objekto išskirtinių savybių 
\cite[37]{cs-beyond-object-oriented-programming} bei
populiariausių objektinių programavimo kalbų kūrėjai nurodo
dar ir tai, kad kiekvienas objektas turi unikalią tapatybę.
Taigi apibendrinant, objektai yra esybės, kurios:
\begin{enumerate}
  \item turi unikalią tapatybę;
  \item turi būsenas;
  \item sąveikauja besikeisdamos žinutėmis;
  \item reaguoja į žinutes įvykdydamos tam tikrą procedūrą.
\end{enumerate}

Siekiant išskirti
\gls{object-oriented-programming-language}{objektines programavimo
kalbas} iš programavimo kalbų, kuriomis įmanoma programuoti
objektiškai (pavyzdžiui, tarę, kad programavimo kalbos \plangname{Modula-2}
modulis yra objektas, su ja galėtume programuoti objektiškai), yra
reikalaujama, kad programavimo kalba, kurią vadiname objektine
palaikytų objektinį programavimą kalbinėmis priemonėmis. 1987
metais ACM OOPSLA konferencijos metu paskelbtame „Orlando
susitarime“ \en{„The Treaty of Orlando“}
\cite{Lieberman:1987:TO:62139.62144} nurodoma, kad esminis
objektinių programavimo kalbų bruožas yra 
\gls{object-description-sharing-mechanism}{dalinimosi \en{sharing}
mechanizmas} – galimybė perpanaudoti esamų objektų apibrėžimų dalis
apibrėžiant naujus objektus.
\gls{object-oriented-programming-language}{Objektinės programavimo kalbos}
tai gali įgyvendinti panaudodamos \gls{delegation}{delegavimo} arba
\gls{class-inheritance}{paveldėjimo}
mechanizmus.\NFIXME{Ar reikia pavyzdžio su Python, kuris parodo, jog
paveldėjimas ir delegavimas yra tas pats
\cite[1]{Stein:1987:DI:38807.38820}?
(Jei $a$ yra klasės $A$ tipo
objektas, o klasė $A$ paveldi iš $B$, tai delegavimo ryšys būtų:
$a \to A \to B$, o \varname{self} yra tiesiog paprasčiausias
argumentas.)}
Šis mechanizmas yra itin svarbus tuo, kad leidžia objektus apibrėžti
palaipsniui. Taigi, \plangname{Modula-2} nėra objektinė programavimo
kalba, nes jos modulis negali automatiškai „perimti“ dalies kito modulio
veiksenos.

Be šios „Orlando susitarime“ nurodytos savybės, dauguma
objektinių programavimo kalbų turi ir daugiau bendrų bruožų,
kuriuos išskyrė \cite[225-227]{types-and-programming-languages}:
\begin{description}
  \item[\gls{dynamic-dispatch}{dinaminis susiejimas}] –
    objekto reakcija į gautą žinutę yra nustatoma vykdymo metu;
  \item[\gls{encapsulation}{uždarumas}] – vidinė objekto
    struktūra yra slepiama;
  \item[\gls{subtyping}{potipiai}] – kai norime pasinaudoti objektu, 
    tai mums rūpi tik jo sąsaja ir mes galime naudoti objektą $I$ vietoj
    objekto $J$, jei objekto $J$ sąsaja yra objekto $I$ sąsajos poaibis;
  \item[\gls{object-description-sharing-mechanism}{paveldėjimas,
    pavedimas}] – galimybė perpanaudoti jau egzistuojantį kodą: tai
    gali būti pasiekiama objektų kūrimui naudojant klases, kurios
    gali paveldėti kai kurias savybes iš tėvinių klasių, arba
    naudojant žinučių persiuntimą;
  \item[\gls{open-recursion}{atvira rekursija}] –
    (dažniausiai jis vadinamas „this“, arba „self“)
    egzistavimas, kuriuo pasinaudojant galima kreiptis į kitus to
    paties objekto metodus.
\end{description}
Nors paminėtos savybės ir yra bendros populiariausioms objektinėms
programavimo kalboms, bet jos jas įgyvendina įvairiais būdais,
dėl ko jos įgauna savitumų, kurie daro įtaką šių programavimo
kalbų galimoms pritaikymo sritims.

\section{Objektinių kalbų rūšys}

Objektinės programavimo kalbos yra naudojamos įvairioms situacijoms:
tiek eksperimentiniam, tiriamajam bei prototipiniam programavimui, kai
dalykinė sritis yra prastai žinoma ir yra svarbus lankstumas, tiek
jau produkcijai skirtoms sistemoms, kurios palyginus
lėtai keičiasi ir kurių patikimumas yra itin
svarbus\cite{Lieberman:1987:TO:62139.62144}. Pagal
\cite{Lieberman:1987:TO:62139.62144} pirmuoju atveju lankstumą
suteikia galimybės:
\begin{itemize}
  \item modifikuoti atskiro objekto veikseną – pavyzdžiui, vienam
    konkrečiam objektui pridėti naują metodą;
  \item keisti \gls{object-description-sharing-mechanism}{dalinimosi
    mechanizmo} ryšius vykdymo metu – pavyzdžiui, pakeisti tėvinį
    objektą;
  \item išreikštinai nurodyti
    \gls{object-description-sharing-mechanism}{dalinimosi mechanizmo}
    ryšius – pavyzdžiui, kokios žinutės kokiam objektui yra
    persiunčiamos.
\end{itemize}
Šiomis savybėmis pasižymi
\gls{prototype-based-programming-language}{prototipinės objektinės
programavimo kalbos}, iš kurių, turbūt, populiariausia atstovė yra
\plangname{JavaScript}.

Kuriamų sistemų patikimumą padeda užtikrinti tokios programavimo
kalbos savybės, kaip:
\begin{itemize}
  \item galimybė užtikrinti, kad visi konkrečiai grupei priklausantys
    objektai elgsis vienodai;
  \item draudimas keisti objekto apibrėžimą po jo sukūrimo – tai
    gali būti naudinga, pavyzdžiui, tuo, kad mes esame garantuoti,
    jog objektas tenkina tą patį kontraktą, kurį jis tenkino iš
    karto po sukūrimo;
  \item vieningas \gls{object-description-sharing-mechanism}{
    apibrėžimų dalinimosi mechanizmas}, kuris palengvina supratimą
    kaip veikia sistema.
\end{itemize}
Šiomis savybėmis pasižymi
\gls{class-based-programming-language}{klasinės objektinės
programavimo kalbos}, tokios, kaip \plangname{Java}, \plangname{C++},
\plangname{Python}, \plangname{Ruby}.

Kaip pavyzdys situacijos, kur prototipinės objektinės programavimo
kalbos savybės yra naudingesnės nei klasinės objektinės, galėtų
būti natūralios žmonių kalbos sintezatoriaus kūrimas. Tarkime, kad
mums reikia galimybės turint veiksmažodžio bendratį susigeneruoti
visas jo formas. Naudodami klasinę objektinę programavimo kalbą
kiekvienai veiksmažodžių rūšiai galėtume susikurti po klasę,
kuri gavusi bendratį mokėtų sugeneruoti tos rūšies veiksmažodžių
formas. Problema ta, kad natūralios žmonių kalbos pasižymi didele
išimčių gausa ir šiuo atveju kiekvienai išimčiai irgi reikėtų
sukurti po klasę. Jei naudotume prototipinę objektinę programavimo
kalbą, tai kiekvieną kartą pridėdami po naują veiksmažodžio
objektą, jam kaip prototipą galėtume nurodyti panašiausią į jį ir
tereikėtų perrašyti tik jų skirtumus.

Pagal \cite[2]{Madsen:1990:STO:97946.97964} programavimo kalbų
lankstumą dar padidina \gls{weakly-typed-programming-language}{silpna
tipizacija} bei \gls{dynamically-typed-programming-language}{dinaminis
tipų tikrinimo mechanizmas}. Tuo tarpu
\gls{strongly-typed-programming-language}{stipri tipizacija} bei
\gls{statically-typed-programming-language}{statinis tipų tikrinimo
mechanizmas} leidžia nemažai klaidų surasti dar programos kompiliavimo
stadijoje, kas padidina programos patikimumą. Taigi siekiant kurti
sistemas, kurioms yra itin svarbus jų patikimumas, derėtų naudoti
\gls{statically-typed-programming-language}{statines}
\gls{strongly-typed-programming-language}{stipriai tipizuotas}
\gls{class-based-programming-language}{klasines} programavimo kalbas.
Kadangi šiame darbe yra nagrinėjamos priemonės, kurios pagelbėtų
kuriant verslo palaikymo sistemas, kurioms ir yra itin svarbus
patikimumo kriterijus, tai toliau darbe yra nagrinėjamos tik
\gls{statically-typed-programming-language}{statinės}
\gls{strongly-typed-programming-language}{stipriai tipizuotos}
\gls{class-based-programming-language}{klasinės}
\gls{object-oriented-programming-language}{objektinės} programavimo
kalbos.
