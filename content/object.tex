\chapter{Objektinis programavimas}

Norint atskirti dvi paradigmas, reikia tiksliai apibrėžti kas priklauso
kiekvienai iš jų. Šiame skyriuje bus pabandyta išskirti kas yra
objektinis programavimas apskritai, kokios yra objektinio
rūšys\NTODO{Nagrinėjant Scala yra svarbu nurodyti ant kokio objektinio
ji „stovi“.} bei kam tos rūšys yra reikalingos (kokios yra jų
pritaikymo sritys)\NTODO{Nagrinėjant Scala panagrinėti ar ji keičia
pritaikymo sritį.}.

\section{Objektinio programavimo apibrėžimai}

Objektinis programavimas bendrąja prasme yra programavimas
pasinaudojant objekto abstrakcija. Objektą galima apibrėžti, kaip
\quote{esybę, kuri apjungia duomenų ir procedūrų savybes tam,
kad galėtų atlikti veiksmus ir turėti
būseną}\cite[41]{OOP-themes-and-variations}. Objektinės sistemos veikimas
yra suvokiamas, kaip objektų tarpusavio sąveika: objektai vienas kitam
siuntinėja žinutes, o kiekvienas gavęs žinutę objektas į ją sureaguoja
įvykdydamas atitinkamos procedūros (metodo)
kodą\cite[41]{OOP-themes-and-variations}%
\cite[277]{concepts-in-programming-languages}%
\cite[168]{Wegner:1987:DOL:38807.38823}.
Be būsenos ir veiksenos, prie objekto išskirtinių savybių 
\cite[37]{cs-beyond-object-oriented-programming} bei
populiariausių objektinių programavimo kalbų autoriai nurodo
dar ir tai, kad kiekvienas objektas turi unikalią tapatybę.
Taigi apibendrinant, objektai yra esybės, kurios:
\begin{enumerate}
  \item turi unikalią tapatybę;
  \item turi būsenas;
  \item sąveikauja besikeisdamos žinutėmis;
  \item reaguoja į žinutes įvykdydamos tam tikrą procedūrą.
\end{enumerate}

Siekiant išskirti objektines programavimo kalbas iš programavimo
kalbų, kuriomis įmanoma programuoti objektiškai (pavyzdžiui,
tarę, kad programavimo kalbos Modula-2 modulis yra objektas,
su ja galėtume programuoti objektiškai), yra reikalaujama, kad
programavimo kalba, kurią vadiname objektine palaikytų objektinį
programavimą kalbinėmis priemonėmis. 1987 metais ACM OOPSLA
konferencijos metu paskelbtame „Orlando susitarime“
\en{„The Treaty of Orlando“}
\cite{Lieberman:1987:TO:62139.62144} nurodoma, kad esminis
objektinių programavimo kalbų bruožas yra galimybė perpanaudoti
esamų objektų apibrėžimų dalis apibrėžiant naujus objektus.
Objektinės programavimo kalbos tai įgyvendina panaudodamos
delegavimo \en{delegation} arba klasių paveldėjimo \en{inheritance}
mechanizmus.

objektinių programavimo kalbų bruožas yra tai, kad kaip objektas
reaguos į konkrečią žinutę, yra nusprendžiama vykdymo metu.
Ši savybė vadinama yra vadinama \strong{dinaminiu susiejimu}
\TODO{Patikrinti ar neturėčiau rašyti „sharing mechanism“.}
\en{dynamic dispatch, dynamic binding}. (Šito reikalavimo Modula-2
moduliai netenkina, nes juose visada yra kviečiama ta pati procedūra.)
Be šios pagrindinės savybės, objektinės programavimo kalbos
dažniausiai turi ir daugiau bendrų bruožų, kuriuos išskyrė
\cite[225-227]{types-and-programming-languages}:
\begin{description}
  \item[dinaminis susiejimas] \en{dynamic dispatch, dynamic binding} –
    objekto reakcija į gautą žinutę nustatoma vykdymo metu;
  \item[uždarumas, inkapsuliacija] \en{encapsulation} – vidinė objekto
    struktūra yra slepiama;
  \item[potipiai] \en{subtyping} – kai norime pasinaudoti objektu, 
    tai mums rūpi tik jo sąsaja ir mes galime naudoti objektą $I$ vietoj
    objekto $J$, jei objekto $J$ sąsaja yra objekto $I$ sąsajos poaibis;
  \item[paveldėjimas, pavedimas] \en{inheritance, delegation} – galimybė
    perpanaudoti jau egzistuojantį kodą: tai gali būti pasiekiama
    objektų kūrimui naudojant klases, kurios gali paveldėti kai kurias
    savybes iš tėvinių klasių, arba naudojant žinučių persiuntimą;
  \item[atvira rekursija] \en{open recursion} – specialaus kintamojo
    (dažniausiai jis vadinamas „this“, arba „self“) egzistavimas, kuriuo
    pasinaudojant galima kreiptis į kitus to paties objekto metodus.
    Svarbi savybė yra \en{late-bound}.
\end{description}


\section{Objektinio programavimo rūšys}

\section{Objektinio programavimo rūšių taikymai}

Prototipinė objektinė programavimo kalba (JavaScript) yra tinkamesnė
žmonių kalbos modeliavimui nei klasinė programavimo kalba, nes
žmonių kalboje paprastai yra daugiau išimčių negu taisyklių.

\section{Objektinio programavimo apibrėžimai}

TODO:
\begin{itemize}
  \item Objektinio programavimo apibrėžimas:
    \begin{itemize}
      \item dinaminis susiejimas;
      \item inkapsuliacija;
      \item potipiai;
      \item paveldėjimas, pavedimas;
      \item atvira rekursija.
    \end{itemize}
  \item Objektinio programavimo rūšys ir kur patartina ką naudoti:
    \begin{itemize}
      \item prototipinis objektinis;
      \item dinaminis klasinis objektinis;
      \item statinis klasinis objektinis.
    \end{itemize}
\end{itemize}

\section{Statinis klasinis objektiškai orientuotas programavimas}

TODO: Apibrėžimas ir nurodymas, kad toliau darbe apsiribojama tik juo.
Kaip pagrindinis pavyzdys nagrinėjama Java programavimo kalba.
