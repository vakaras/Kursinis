\chapter{Komponentinės paradigmos apibrėžimas}

\section{Komponentinių programų sistemų inžinerija}

\cite{analytical-study-cbse} pateikti tokie principai:
\begin{description}
  \item[nepriklausomas programinės įrangos kūrimas] \en{independent
    software development} – didelės sistemos yra sujungiamos iš dalių,
    kurias kūrė skirtingi žmonės. Tam, kad palengvinti nepriklausomą
    kūrimą, yra būtina atskirti komponentų kūrėjus nuo jų naudotojų,
    panaudojant nuo realizacijos nepriklausomą sąsają, kuri aprašo
    komponento elgesį;
  \item[perpanaudojamumas] \en{reusability} – 
  \item[programinės įrangos kokybė] \en{software quality} – turi
    būti parodyta, kad komponentas ar sistema elgesi būtent taip, kaip
    yra tikimasi;
  \item[palaikymas] \en{maintainability} – sistema turi būti suprantama
    ir lengvai plečiama.
\end{description}

„Komponentas yra programinės įrangos elementas, kuris prisitaiko prie
programinės įrangos modelio ir gali būti nepriklausomai vystomas
bei įkomponuotas nedarant jame pakeitimų ir laikantis komponavimo
standarto.“\cite[438]{analytical-study-cbse} (Originalas iš
\cite{heineman2001component}.)

\section{Sutvarkyti}

TODO: Apibrėžimas iš UML.

TODO: „Component software: beyond object-oriented programming By
Clemens Szyperski, Dominik Gruntz, Stephan Murer“
\url{http://books.google.com/books?hl=en&lr=&id=U896iwmtiagC&oi=fnd&pg=PR15&dq=object-oriented+programming+dynamic+inheritance&ots=FFWO7xstgM&sig=RhMmypIZ6LsrfHWf4IGB9e3yl4Q#v=onepage&q=object-oriented programming dynamic inheritance&f=false}
